\documentclass[11pt]{exam}
\usepackage[spanish]{babel} % Permite el idioma espa\~nol.
\usepackage[utf8]{inputenc}
\usepackage{amsmath,epsfig}
\usepackage[colorlinks]{hyperref}
\usepackage{minted} 
\usemintedstyle{emacs}
\usepackage{tcolorbox} % colores para el fondo
\definecolor{bg}{rgb}{0.95,0.95,0.95} %color de fondo

\pagestyle{headandfoot}

\begin{document}


\firstpageheadrule
%\firstpagefootrule
%\firstpagefooter{}{Pagina \thepage\ de \pages}{}
\runningheadrule
%\runningfootrule
\lhead{\bf\normalsize Computación Científica\\ Guía 03}
\rhead{\bf\normalsize Cs. Fís., Astro., Geofís. \\ 2024-1}
\chead{\bf\normalsize Depto. de Física \\ Universidad de Concepción}
%\rfoot{\thepage\ / pages}
\cfoot{ }
\lfoot{\tiny GR}
\begin{flushleft}
\vspace{0.2in}
%\hbox to \textwidth{Nombre: \enspace \hrulefill}
%Nombre : \\
\vspace{0.25cm}
\end{flushleft}
%%%%%%%%%%%%%%%%%%%%%%%%%%%%%%%%%%%%%%%%%%

\begin{questions}


\item Vea \href{https://youtu.be/VSH1XpWN1us}{este video} para aprender un poco más sobre el uso de \texttt{|} (ver guía 2). Reproduzca/adapte los ejemplos ahí mostrados.

\item El comando \texttt{echo} despliega en la salida principal (la pantalla) un mensaje de texto indicado.
\begin{parts}
\item Pruebe qué hace el comando 
\begin{minted}[bgcolor=bg]{bash}
echo 'Hola Mundo'
\end{minted}
\item Con la opción \texttt{-e} el comando \texttt{echo} reconoce algunos caracteres especiales, por ejemplo \verb|\n| es reconocido como un salto a una nueva línea. Verifique esto ejecutando
\begin{minted}[bgcolor=bg]{bash}
echo -e 'Hola\nMundo'
\end{minted}
\item Investigue qué otros caracteres especiales son reconocidos por \texttt{echo -e}.
\end{parts}

\item Además de los caracteres de redireccionamiento $>$ y $\vert$, existe $>\,>$, que también redirecciona la salida de un comando a un archivo. La diferencia entre $>$ y $>\,>$ es que en el primer caso el nuevo archivo se crea desde cero (y si ya existe, se reemplaza por el nuevo), mientras que $>>$ agrega la salida al final de un archivo preexistente.

Para ejercitar lo anterior, haga algunas pruebas simples con $>$ y $>\,>$. Luego de esto escriba un comando \texttt{Bash} que junte todos los trozos del texto del Quijote (archivos \texttt{c1.txt} a \texttt{c5.txt}), lo guarde en el nuevo archivo \texttt{Quijote.txt},   y que luego le agregue al final una nueva línea con la palabra `FIN'. Nuevamente, puede realizar todo esto con una única línea de comandos.

\item Usando el comando \texttt{wget} (ver guía 1), descargue el archivo de datos \verb|SCPUnion2.1_mu_vs_z.txt| desde la dirección \url{http://supernova.lbl.gov/Union/figures/SCPUnion2.1_mu_vs_z.txt} alojado en el sitio del \textit{Supernova Cosmology Project} (SCp,
\url{http://supernova.lbl.gov/}). Este archivo contiene datos de las observaciones astronómicas de estrellas Supernovas, que son de gran importancia en Cosmología. Una vez descargado el archivo, despliege su contenido en la consola (usando los comandos \texttt{more}, \texttt{less}, \texttt{cat}, \texttt{head} y \texttt{tail}. ¿En qué se diferencias c/u de estos comandos?).

\item El comando \texttt{cut} selecciona partes del texto en un archivo. Por ejemplo, usando el archivo de datos de Supernovas descargado en el punto anterior, podemos seleccionar (`cortar') sólo la tercera columna de datos con el comando
\begin{minted}[bgcolor=bg]{bash}
cut -s -f 3 SCPUnion2.1_mu_vs_z.txt
\end{minted}
\begin{parts}
\item Verifique que la opción \texttt{-s} sirve para que \texttt{cut} se salte las primeras líneas del archivo, que contienen comentarios y no datos.
\item Verifique que la opción \texttt{-f 3} selecciona la tercera columna (pruebe seleccionando otras columnas).
\end{parts}

\item Use el editor de texto \texttt{nano} en la consola y cree un archivo de texto llamado \texttt{plantilla.tex}, con el siguiente contenido (que usará en la primera práctica de \LaTeX, la próxima semana):

\begin{minted}[bgcolor=bg]{tex}
\documentclass[12pt]{article}
\usepackage[utf8]{inputenc}

\begin{document}
Éste es mi primer documento en \LaTeX
 
\end{document}
\end{minted}
\end{questions}
\end{document} 