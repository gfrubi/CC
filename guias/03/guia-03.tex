\documentclass[11pt]{exam}
\usepackage[activeacute,spanish]{babel} % Permite el idioma espa\~nol.
\usepackage[utf8]{inputenc}
\usepackage{amsmath,epsfig}
%\usepackage{graphdicx}
\usepackage[colorlinks]{hyperref}
\usepackage{minted} 
\usemintedstyle{emacs}
\usepackage{tcolorbox} % colores para el fondo
\definecolor{bg}{rgb}{0.95,0.95,0.95} %color de fondo

\pagestyle{headandfoot}

\begin{document}

\firstpageheadrule
%\firstpagefootrule
%\firstpagefooter{}{Pagina \thepage\ de \pages}{}
\runningheadrule
%\runningfootrule
\lhead{\bf\normalsize Computación Científica\\ Guía 03}
\rhead{\bf\normalsize Cs. Fís., Astro., Geofís. \\ 2020-1}
\chead{\bf\normalsize Depto. de Física \\ Universidad de Concepción}
%\rfoot{\thepage\ / pages}
\cfoot{ }
\lfoot{\tiny GR}
\begin{flushleft}
\vspace{0.2in}
%\hbox to \textwidth{Nombre: \enspace \hrulefill}
%Nombre : \\
\vspace{0.25cm}
\end{flushleft}
%%%%%%%%%%%%%%%%%%%%%%%%%%%%%%%%%%%%%%%%%%


\begin{questions}
\item Trabaje con el archivo con el código \LaTeX\ que creó al final de la guía pasada (\texttt{plantilla.tex}) y compílelo con el comando \texttt{pdflatex plantilla.tex}. Si todo sale bien, debe generar directamente un archivo .pdf con su primer trabajo en \LaTeX.
Abra el archivo .pdf para visualizar el resultado.

\item Para saciar su infinita curiosidad, mire (en la consola!) el contenido de los archivos auxiliares generados (.aux y .log). Luego de esto, borre todos los archivos generados por la compilación.

\item Usando el comando \texttt{cp} haga dos copias de su archivo \texttt{plantilla.tex} con nombres \texttt{test-01.tex} y \texttt{test-02.tex}. Guarde el archivo \texttt{plantilla.tex} en algún lugar seguro, le servirá en el futuro.

\item Agregue a su archivo \texttt{test-01.tex} algunas secciones y texto que involucre caracteres latinos, usando \verb|\'a, \'e, \'i, \'o, \'u, \~n y ?`|, que generan á, é, í, ó, í, \~n, y ?`, respectivamente

\item Ahora agregue el siguiente código en alguna parte de su documento:

\begin{minted}[bgcolor=bg]{tex}
\begin{quote}
``El primer principio es que no te debes enga\~nar a ti mismo - y t\ú eres 
la persona que m\ás f\ácilmente te enga\~na. Así que hay que tener mucho 
cuidado con eso. Una vez que no te enga\~nas a ti mismo, es f\ácil que no 
enga\~nes a los otros científicos''. \texttt{Richard Feynman}.
\end{quote}
\end{minted}

Esto introduce el texto dentro del entorno \texttt{quote}, que es apropiado para citar frases célebres de algún personaje importante. Vea cómo luce el resultado en su archivo .pdf.

\texttt{Ojo!} Existen tres tipos de comillas: 	las comillas ``simples'' ('\,), las comillas ``dobles'' ("\,), y las comillas ``diagonales hacia la derecha'' (`). Éstas se obtienen con combinaciones distintas de teclas (que var'ian de teclado en teclado!). Las comillas usadas en el ejemplo del entorno \texttt{quote} son dos comillas diagonales al comienzo y dos comillas simples al final de la frase.


\item Cambie el tipo de entorno usado en el punto anterior desde \texttt{quote}, para que ahora sea un entorno \texttt{center}, \texttt{flushleft}, \texttt{flushright} y finalmente \texttt{sloppypar}. En cada caso, vea cómo esto afecta al resultado final.

\item Lea el pdf de la \href{https://github.com/gfrubi/CC/blob/master/LaTeX/clases-LaTeX.pdf}{presentación de \LaTeX} usada en clases, hasta la página 23 (``Espa\~nol y \LaTeX'').

\item Descargue el archivo modelo \href{https://github.com/gfrubi/CC/blob/master/guias/03/articulo.pdf}{articulo.pdf} y ábralo para ver qué contiene. 


\item Edite \texttt{test-02.tex} para que al compilarlo se reproduzca lo más fielmente posible el contenido del model en el archivo \texttt{articulo.pdf} (secciones, subsecciones, listas, texto, etc.).

\item En el archivo \texttt{test-02.tex} realice las siguientes modificaciones y observe qué efecto tiene cada una de ellas en el .pdf final.
\begin{parts}
\item Agregue el comando \verb|\tableofcontents| en la l'inea siguiente a \verb|\begin{document}|. No olvide compilar dos veces para ver el efecto de este cambio!.
\item Agregue el comando \verb|\usepackage[spanish]{babel}| en la segunda l'inea del código, es decir, en la l'inea siguiente a \verb|\documentclass[12pt]{article}|.
\item Agregue la opción \verb|twocolumn| a la declaración de clase de la primera l'inea, es decir, transfórmela en \verb|\documentclass[12pt,twocolumn]{article}|.
\item Finalmente, modifique la opción \verb|12pt| en la primera l'inea, reemplazándola por \verb|10pt|.
\end{parts}

\item \LaTeX\ es un mundo vasto, bello y desconocido, en el que se pueden seguir aprendiendo y desarrollando nuevos aspectos constantemente. Para explorar un poco más, descargue y dé un vistazo al tutorial ``\textit{La introducción no-tan-corta a \LaTeX 2e}'' (2014), de Tobias Oetiker, Hubert Partl, Irene Hyna y Elisabeth Schlegl, disponible en \url{http://www.ctan.org/tex-archive/info/lshort/spanish}. Note que en la subcarpeta \texttt{fuente/src} del link anterior está disponible el código \LaTeX que genera este documento.

\item Finalmente, otra muy buena referencia para aprender y/o consultar sobre \LaTeX\ es el libro ``\textit{Edición de Textos Cient'ificos en \LaTeX: Composición, Dise\~no Editorial, Gráficos, Inkscape, Tikz y Presentaciones Beamer}'' (2da edición, actualización Enero 2021), de Alexánder Borbón y Walter Mora , disponible en \url{https://tecdigital.tec.ac.cr/revistamatematica/Libros/LaTeX/}. Descargue este libro, mire qué contiene y guádelo para refencia futura.
\end{questions}
\end{document} 