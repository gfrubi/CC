\documentclass[11pt]{exam}
\usepackage[spanish]{babel} % Permite el idioma español.
\usepackage[utf8]{inputenc}
\usepackage{amsmath}
\usepackage[colorlinks]{hyperref}

\usepackage{minted} 
\usemintedstyle{emacs}
\usepackage{tcolorbox} % colores para el fondo
\definecolor{bg}{rgb}{0.95,0.95,0.95} %color de fondo

\pagestyle{headandfoot}
\spanishdecimal{.}

\begin{document}

\firstpageheadrule
%\firstpagefootrule
%\firstpagefooter{}{Pagina \thepage\ de \pages}{}
\runningheadrule
%\runningfootrule
\lhead{\bf\normalsize Computación Científica\\ Guía 10}
\rhead{\bf\normalsize Cs. Fís., Astro., Geofís. \\ 2023-1}
\chead{\bf\normalsize Depto. de Física \\ Universidad de Concepción}
%\rfoot{\thepage\ / pages}
\cfoot{ }
\lfoot{\tiny GR}
\begin{flushleft}
\vspace{0.2in}
%\hbox to \textwidth{Nombre: \enspace \hrulefill}
%Nombre : \\
\vspace{0.25cm}
\end{flushleft}
%%%%%%%%%%%%%%%%%%%%%%%%%%%%%%%%%%%%%%%%%%

\begin{questions}
\item Escriba un programa que imprima en pantalla si un número entero positivo ingresado desde el teclado es o no un número primo. Utilice un ciclo \texttt{while}.

\item Considere una pelota que se lanza verticalmente desde una altura inicial $z_0=3\,\rm m$, con una velocidad inicial $v_0 = 1 \,\rm m/s$ hacia arriba. Calcule e imprima en pantalla los valores de la altura
\begin{equation}
 z(t) = z_0 + v_0 t -\frac{g}{2}t^2,
\end{equation} 
y el correspondiente tiempo $t$ para $t=0,0.01,0.02,\dots$ hasta (justo antes) que la pelota llegue al suelo (es decir, llegue a $z=0$). Utilice un ciclo \texttt{while} y considere $g=9.8\,\rm m/s^2$.

\item Vea \href{https://udec.instructure.com/courses/40179/pages/funciones?module_item_id=1465539}{este video} disponible en Canvas, en el que se explica la definición de \textbf{funciones} en Python. Reproduzca y verifique todos los ejemplos ahí descritos.

\item Modifique el código que creó para calcular el factorial de un número entero (guía 09) para que ahora se defina una función \texttt{mifactorial}, de modo que el factorial de $n$ se pueda luego llamar como \texttt{mifactorial(n)}.

\item La exponencial $e^x$ de un número real $x$ puede ser calculada usando la siguiente serie
\begin{equation}\label{e}
e^x = \sum_{n=0}^\infty \frac{x^n}{n!}=1 + \frac{x}{1!} + \frac{x^2}{2!} + \frac{x^3}{3!} + \frac{x^4}{4!} + \cdots
\end{equation}
Reutilizando su código para calcular factoriales, escriba un programa que pregunte al usuario el valor de $x$ y calcule e imprima el valor de $e^x$, usando la expresión \eqref{e}. Para el cálculo considere 100 términos en la suma \eqref{e}, es decir, que el programa calcule la suma hasta el término $x^{99}/{99!}$. 

\textbf{Nota 1}: En el caso $n=0$, se define el factorial de $0$ igual al valor $1$, es decir, $0! :=1$.

\textbf{Nota 2}: El ``truncar'' la serie (es decir, evaluarla hasta cierto número de términos) tiene como consecuencia que el valor calculado es sólo una \textit{aproximación} del valor exacto ($e^x$). Esta aproximación es mejor si se incluyen más términos.

\item Utilice el programa que acaba de escribir para calcular (una aproximación d)el valor de $e$ (el número de Euler). Compare su resultado con el valor listado en este \href{https://es.wikipedia.org/wiki/N\%C3\%BAmero_e}{artículo de wikipedia}.

\item Escriba un programa que evalúe (una aproximación de) el número $\pi$. Para esto, use la siguiente expresión en serie (desarrollada por el gran \href{https://es.wikipedia.org/wiki/Leonhard_Euler}{Leonhard Euler}),
\begin{equation}
\pi = \sum_{n=0}^{\infty}\frac{2^{n+1}(n!)^2}{(2n+1)!}=\left[2^1\frac{(0!)^2}{1!} + 2^2\frac{(1!)^2}{3!} + 2^3\frac{(2!)^2}{5!} + 2^4 \frac{(3!)^2}{7!} + \cdots \right].
\end{equation}
Lo anterior es un ejemplo de un método con el que se puede calcular el valor de $\pi$, con precisión cada vez mayor al agregar más y más términos. Dado que $\pi$ es un número irracional, sólo se conoce su valor (calculado con métodos similares) hasta un cierto número de decimales. El record actual lo tiene Shigeru Kondo, quien logró calcular $\pi$ con $10 000 000 000 000$ decimales!. Compare el valor que usted obtenido con el listado en este \href{https://es.wikipedia.org/wiki/N\%C3\%BAmero_pi}{artículo de wikipedia}.

\end{questions}
\end{document} 