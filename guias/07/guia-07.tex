\documentclass[11pt]{exam}
\usepackage[spanish]{babel} % Permite el idioma español.
\usepackage[utf8]{inputenc}
\usepackage{amsmath}
\usepackage[colorlinks]{hyperref}

\usepackage{minted} 
\usemintedstyle{emacs}
\usepackage{tcolorbox} % colores para el fondo
\definecolor{bg}{rgb}{0.95,0.95,0.95} %color de fondo

\pagestyle{headandfoot}

\begin{document}

\firstpageheadrule
%\firstpagefootrule
%\firstpagefooter{}{Pagina \thepage\ de \pages}{}
\runningheadrule
%\runningfootrule
\lhead{\bf\normalsize Computación Científica\\ Guía 07}
\rhead{\bf\normalsize Cs. Fís., Astro., Geofís. \\ 2024-1}
\chead{\bf\normalsize Depto. de Física \\ Universidad de Concepción}
%\rfoot{\thepage\ / pages}
\cfoot{ }
\lfoot{\tiny GR}
\begin{flushleft}
\vspace{0.2in}
%\hbox to \textwidth{Nombre: \enspace \hrulefill}
%Nombre : \\
\vspace{0.25cm}
\end{flushleft}
%%%%%%%%%%%%%%%%%%%%%%%%%%%%%%%%%%%%%%%%%%

\begin{questions}
%\item Vea el video resumen sobre cómo incluir figuras/gráficos en un documento \LaTeX, disponible en Canvas \href{https://udec.instructure.com/courses/40179/pages/figuras?module_item_id=1465511}{aquí}.
\item En algún archivo .tex que ya tenga hecho, inserte el escudo de la UdeC, contenido en el archivo \url{http://www.udec.cl/normasgraficas/sites/default/files/marcaderecha.png}:
\begin{figure}[h!]
\begin{center}
\includegraphics[width=4cm]{marcaderecha.png}
\end{center}
\caption{Marca alineación derecha, Formato PNG.}
\label{fig:escudo}
\end{figure}

Vea \href{http://www.udec.cl/normasgraficas/node/7}{esta} página para conocer otras variaciones oficiales del escudo de la UdeC. 
Puede encontrar más información acerca de las normas gráficas de nuestra Universidad en \href{http://www.udec.cl/normasgraficas}{esta} página.

\item Ahora su misión es escribir su primer ``artículo científico completo'' en \LaTeX, que debe reproducir lo más fielmente posible \href{https://github.com/gfrubi/CC/blob/master/guias/07/ejemplo-articulo.pdf}{este ejemplo}. Para eso, use todo lo aprendido (en particular, use referencias cruzadas a las ecuaciones, tablas, figuras y referencias). El archivo .pdf de la figura lo puede descargar desde \href{https://github.com/gfrubi/CC/blob/master/guias/07/fig-ajuste-lineal.pdf}{aquí}.

El título, autor(a), y resumen puede ser incorporados usando los siguientes comandos después del conocido \verb|\documentclass|:

\begin{minted}[bgcolor=bg]{tex}
\title{Escribiendo mi primer artículo con formato Científico, en \LaTeX}
\author{Su nombre (autor(a))}

\begin{document}

\maketitle
\begin{abstract}
Este es el resumen del artículo...
\end{abstract}
\end{minted}

\item Vea el video en el que se explica el uso de la clase Beamer para la creación de presentaciones, disponible en Canvas \href{https://udec.instructure.com/courses/40179/pages/presentaciones-en-beamer}{aquí}.
\item Lea las páginas 66 a 71 del \href{https://udec.instructure.com/courses/40179/pages/latex-pdf-presentacion?module_item_id=1465499}{pdf de clases}.
\item Descargue el archivo de ejemplo \href{https://raw.githubusercontent.com/gfrubi/CC/master/LaTeX/ejemplo-beamer.tex}{ejemplo-beamer.tex} y estudie su contenido. Asegúrese de entender qué hace cada línea de código.
\item Explore cómo lucen los distintos temas disponibles en Beamer, cambiando la opción \verb|\usetheme{}|. Puede visualizar los distintos temas en \href{https://deic.uab.cat/~iblanes/beamer_gallery/index_by_theme.html}{esta página}.

\item Además de los temas anteriores, es posible cambiar el esquema de colores usado, con el comando \verb|\usecolortheme|. Por ejemplo, incluyendo el comando 
\begin{minted}[bgcolor=bg]{tex}
\usecolortheme{crane}
\end{minted}
en el preámbulo se selecciona el esquema de colores ``\texttt{crane}''. Los esquemas de colores disponibles son: \texttt{albatross, beaver, beetle, crane, default, dolphin, dove, fly, lily, orchid, rose, seagull, seahorse, sidebartab, structure, whale, wolverine}. Puede visualizar estos esquemas de colores en \href{https://deic.uab.cat/~iblanes/beamer_gallery/index_by_color.html}{esta página}. Pruebe algunos de estos colores y seleccione el que más le guste.

\item Adapte el archivo de ejemplo para crear su propia, única y espectacular presentación. Ésta debe incluir \textit{como mínimo}: el logo de la UdeC en la portada, la tabla de contenidos, una hermosa expresión matemática (entorno \texttt{equation}), una ecuación alineada (entorno \texttt{eqnarray} o \texttt{align}), una tabla y alguna linda figura.
\item Comparta una copia del archivo pdf su presentación en el \href{https://teams.microsoft.com/l/channel/19\%3A687f654a686c467bb9f41e0970b179a6\%40thread.tacv2/pr\%C3\%A1cticas?groupId=adc7e83d-1b47-4662-9902-fedf7923a4fb&tenantId=56582b9e-8824-49d0-a665-cd328c0e004a}{canal de prácticas} del grupo de Teams del curso.
\end{questions}
\end{document} 