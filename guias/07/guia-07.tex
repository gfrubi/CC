\documentclass[11pt]{exam}
\usepackage[spanish]{babel} % Permite el idioma español.
\usepackage[utf8]{inputenc}
\usepackage{amsmath}
\usepackage[colorlinks]{hyperref}

\usepackage{minted} 
\usemintedstyle{emacs}
\usepackage{tcolorbox} % colores para el fondo
\definecolor{bg}{rgb}{0.95,0.95,0.95} %color de fondo

\pagestyle{headandfoot}

\begin{document}

\firstpageheadrule
%\firstpagefootrule
%\firstpagefooter{}{Pagina \thepage\ de \pages}{}
\runningheadrule
%\runningfootrule
\lhead{\bf\normalsize Computación Científica\\ Guía 07}
\rhead{\bf\normalsize Cs. Fís., Astro., Geofís. \\ 2023-1}
\chead{\bf\normalsize Depto. de Física \\ Universidad de Concepción}
%\rfoot{\thepage\ / pages}
\cfoot{ }
\lfoot{\tiny GR}
\begin{flushleft}
\vspace{0.2in}
%\hbox to \textwidth{Nombre: \enspace \hrulefill}
%Nombre : \\
\vspace{0.25cm}
\end{flushleft}
%%%%%%%%%%%%%%%%%%%%%%%%%%%%%%%%%%%%%%%%%%

\begin{questions}
\item Vea el video en el que se explica el uso de la clase Beamer para la creación de presentaciones, disponible en Canvas \href{https://udec.instructure.com/courses/40179/pages/presentaciones-en-beamer}{aquí}.
\item Lea las páginas 66 a 71 del \href{https://udec.instructure.com/courses/40179/pages/latex-pdf-presentacion?module_item_id=1465499}{pdf de clases}.
\item Descargue el archivo de ejemplo \href{https://raw.githubusercontent.com/gfrubi/CC/master/LaTeX/ejemplo-beamer.tex}{ejemplo-beamer.tex} y estudie su contenido. Asegúrese de entender qué hace cada línea de código.
\item Explore cómo lucen los distintos temas disponibles en Beamer, cambiando la opción \verb|\usetheme{}|.
\item Adapte el archivo de ejemplo para crear su propia, única y espectacular presentación. Ésta debe incluir \textit{como mínimo}: el logo de la UdeC en la portada, la tabla de contenidos, una hermosa expresión matemática (entorno \texttt{equation}), una ecuación alineada (entorno \texttt{eqnarray} o \texttt{align}), una tabla y alguna linda figura.
\item Comparta una copia del archivo pdf su presentación en el \href{https://teams.microsoft.com/l/channel/19\%3a71886676efd74657a9520b2c075c7c3f\%40thread.tacv2/02\%2520Pr\%25C3\%25A1cticas?groupId=221daffe-974a-46b7-b9ef-311b29bce6b4&tenantId=56582b9e-8824-49d0-a665-cd328c0e004a}{canal de prácticas} del grupo de Teams del curso.
\end{questions}
\end{document} 