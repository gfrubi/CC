\documentclass[11pt]{exam}
\usepackage[activeacute,spanish]{babel} % Permite el idioma espa\~nol.
\usepackage[utf8]{inputenc}
\usepackage{amsmath,epsfig}
\usepackage[colorlinks]{hyperref}
\usepackage{minted} 
\usemintedstyle{emacs}
\usepackage{tcolorbox} % colores para el fondo
\definecolor{bg}{rgb}{0.95,0.95,0.95} %color de fondo

\pagestyle{headandfoot}

\begin{document}


\firstpageheadrule
%\firstpagefootrule
%\firstpagefooter{}{Pagina \thepage\ de \pages}{}
\runningheadrule
%\runningfootrule
\lhead{\bf\normalsize Computación Científica\\ Guía 02}
\rhead{\bf\normalsize Cs. Fís., Astro., Geofís. \\ 2024-1}
\chead{\bf\normalsize Depto. de Física \\ Universidad de Concepción}
%\rfoot{\thepage\ / pages}
\cfoot{ }
\lfoot{\tiny GR}
\begin{flushleft}
\vspace{0.2in}
%\hbox to \textwidth{Nombre: \enspace \hrulefill}
%Nombre : \\
\vspace{0.25cm}
\end{flushleft}
%%%%%%%%%%%%%%%%%%%%%%%%%%%%%%%%%%%%%%%%%%

\begin{questions}

\item Conozca el comando \texttt{cat}, que con\textbf{cat}ena (es decir, une) archivos y los despliega en la pantalla (la salida estándar).
\begin{parts}
\item  Infórmese de los aspectos generales de este comando desplegando el manual, con 

\begin{minted}[bgcolor=bg]{bash}
man cat
\end{minted}



\item En la consola, cree dos archivos \textbf{test01.txt} y \textbf{test02.txt} usando el editor \texttt{nano} e introduzca en ellos alg'un texto interesante.

\item Ejecute ahora el comando 

\begin{minted}[bgcolor=bg]{bash}
cat test01.txt test02.txt
\end{minted}

Observe cómo los dos archivos se despliegan en pantalla, uno luego del otro.

\item ¿Cuál es la diferencia del comando anterior respecto a 

\begin{minted}[bgcolor=bg]{bash}
cat *.txt
\end{minted}

Compruebe su respuesta con un ejemplo.
\end{parts}

\item Descargue el archivo \href{https://github.com/gfrubi/CC/raw/master/guias/01/Quijote.tar.gz}{Quijote.tar.gz} (desde la dirección \url{https://github.com/gfrubi/CC/raw/master/guias/01/Quijote.tar.gz}) a su computador local. Este archivo contiene (en forma comprimida) cinco archivos correspondientes a algunas secciones iniciales de la obra ``Don Quijote de la Mancha"\, de Miguel de Cervantes. Abra una consola e ingrese (usando el comando \texttt{cd}) a la carpeta donde se encuentra el archivo descargado y realice las siguientes tareas:
\begin{parts}
\item Descomprima el archivo, usando el comando 

\begin{minted}[bgcolor=bg]{bash}
tar -xf Quijote.tar.gz
\end{minted}

Esto creará cinco archivos con extensión \textbf{.txt}. 
\item Infórmese de los aspectos generales del comando \texttt{tar}, desplegando el manual del comando \texttt{tar} (ingrese \texttt{man tar}).
\item Despliege el contenido de cada archivo usando el comando \texttt{more}.
\item Una los archivos (en orden!) y guarde el resultado en el archivo \textbf{Quijote.txt}. Para esto, la opción \texttt{>} para redireccionar la salida de \texttt{cat} al archivo correspondiente.
\end{parts}

\item Conozca el comando \texttt{grep}, que busca (conjuntos de) palabras dentro de archivos de texto. 
\begin{parts}
\item Busque información en internet sobre este comando. Por ejemplo, realice la b'usqueda ``comando grep linux'' en Google. 'Este es un m'etodo que usualmente permite encontrar rápidamente información 'util. Por ejemplo, el correspondiente \href{https://es.wikipedia.org/wiki/Grep}{artículo de Wikipedia} es un buen punto de partida.

\item Usando \texttt{grep}, busque todas las ocurrencias de la palabra \textbf{Mancha} en cada uno de los cinco archivos originales del Quijote (\textbf{c1.txt} \dots \textbf{c5.txt})

\item Agregue la opción \texttt{-n} al comando \texttt{grep} usado anteriormente (es decir, use \texttt{grep -n}). ¿Qu'e efecto tiene sobre el resultado?
\end{parts}

\item Puede escribir y ejecutar una secuencia de comandos almacenados en un ``\textit{script}'' (archivo de comandos).
\begin{parts}
\item Para este ejemplo, cree una nueva carpeta y cree en ella una copia del archivo comprimido \textbf{Quijote.tar.gz}.

\item En esta carpeta, cree un archivo con nombre \textbf{mi-script.sh} (la extensión .sh es opcional, pero conveniente). En este archivo escriba una variación de los  comandos antes usados, cada uno en una l'inea separada, que realicen las siguientes acciones: 1) descomprima el archivo \textbf{Quijote.tar.gz}, 2) Una los archivos .txt y crea el archivo \textbf{Quijote.txt} y 3) busca todas las ocurrencias de la palabra \textbf{batalla} en este archivo y 4) crea el archivo \textbf{resultado.txt} con el resultado de la b'usqueda.

\item Ejecute su script (es decir, los comandos que contiene el archivo \textbf{mi-script.sh} en el orden en que están escritos), ejecutando 

\begin{minted}[bgcolor=bg]{bash}
bash mi-script.sh
\end{minted}
\end{parts}


\item Entre las funcionalidades más interesantes y útiles del uso de comandos en Linux es el de las ``tuberías'' (``pipes"\ en inglés), que permiten redireccionar el resultado de un comando (su ``salida") a (la ``entrada" de) otro. Para esto, usamos el símbolo \texttt{|}. Por ejemplo, el comando
\begin{minted}[bgcolor=bg]{bash}
history | grep cd
\end{minted}
redirige la salida del comando \texttt{history} (que despliega el historial de comandos que ha ejecutado) al comando \texttt{grep} que entonces buscará la parabra \texttt{cd} en el texto generado por \texttt{history}. Compruebe lo anterior ejecutando el comando \texttt{history} por separado y luego el comando compuesto mostrado arriba.

\item También es posible redireccionar la salida de un comando a un archivo, usando el caracter \texttt{>}. Verifique que el comando
\begin{minted}[bgcolor=bg]{bash}
history > historial.txt
\end{minted}
redirecciona la salida del comando \texttt{history} a un (nuevo) archivo con nombre \texttt{historial.txt}. ¿Qué ocurre si ese archivo ya existía?.

\item Borre el archivo creado en el punto anterior con el comando \texttt{rm}, ejecutando
\begin{minted}[bgcolor=bg]{bash}
rm historial.txt
\end{minted}
Aprenda un poco más sobre las opciones disponibles para este comando llamando a su manual en la consola, por medio de
\begin{minted}[bgcolor=bg]{bash}
man rm
\end{minted}


\item Usando comandos Bash, encuentre todas líneas del texto del Quijote en las que se menciona la palabra ``Mancha'' (primera letra en mayúsculas!), y almacene el texto de estas líneas en el archivo \textbf{Manchas.txt}.

\item En Linux, los archivos cuyo nombre comienzan con un punto (es decir, el caracter ``.'') son considerados como ``archivos ocultos'', que por defecto no son listados por el comando \texttt{ls} (ni por los administradores gráficos de archivos). Aparte de esta característica, son archivos normales, pero que son usados como archivos de configuración del sistema o de algunos programas. La opción \texttt{-a} del comando \texttt{ls} es usada para listar todos los archivos en una carpeta, incluyendo los ocultos. Usando comandos \texttt{Bash}, realice las siguientes tareas
\begin{parts}
\item Liste todos los archivos de la carpeta en la que está trabajando, incluyendo los ocultos.
\item Filtre la lista de archivos, usando el comando \texttt{grep}, para listar sólo los archivos ocultos.
\item Guarde la lista obtenida en el punto anterior en un nuevo archivo con nombre \texttt{ocultos.txt}.
\item Cambie el nombre a algún archivo que haya creado para convertirlo en archivo oculto. Verifique que ahora \texttt{ls} no lo lista por defecto.
\end{parts}

\item Investigue qué efecto tiene la opción \texttt{-o} en el comando \texttt{grep} y ejecute algunos comandos de prueba para verificar su funcionamiento. Luego de esto, ejecute comandos \texttt{Bash} que cuenten cuántas veces se repite la letra ``a'' en el extracto de el Quijote que usó en esta guía (archivos \texttt{c1.txt} a \texttt{c5.txt}). Si usó las característias de redireccionamiento de \texttt{Bash} (el caracter \texttt{|}) debiese realizar esta tarea con una única línea de comando.

%\item Vea \href{https://youtu.be/VSH1XpWN1us}{este video} para aprender un poco más sobre el uso de \texttt{|}. Reproduzca/adapte los ejemplos ahí mostrados.
%
%\item El comando \texttt{echo} despliega en la salida principal (la pantalla) un mensaje de texto indicado.
%\begin{parts}
%\item Pruebe qué hace el comando 
%\begin{minted}[bgcolor=bg]{bash}
%echo 'Hola Mundo'
%\end{minted}
%\item Con la opción \texttt{-e} el comando \texttt{echo} reconoce algunos caracteres especiales, por ejemplo \verb|\n| es reconocido como un salto a una nueva línea. Verifique esto ejecutando
%\begin{minted}[bgcolor=bg]{bash}
%echo -e 'Hola\nMundo'
%\end{minted}
%\item Investigue qué otros caracteres especiales son reconocidos por \texttt{echo -e}.
%\end{parts}
%
%\item Además de los caracteres de redireccionamiento $>$ y $\vert$, existe $>\,>$, que también redirecciona la salida de un comando a un archivo. La diferencia entre $>$ y $>\,>$ es que en el primer caso el nuevo archivo se crea desde cero (y si ya existe, se reemplaza por el nuevo), mientras que $>>$ agrega la salida al final de un archivo preexistente.
%
%Para ejercitar lo anterior, haga algunas pruebas simples con $>$ y $>\,>$. Luego de esto escriba un comando \texttt{Bash} que junte todos los trozos del texto del Quijote (archivos \texttt{c1.txt} a \texttt{c5.txt}), lo guarde en el nuevo archivo \texttt{Quijote.txt},   y que luego le agregue al final una nueva línea con la palabra `FIN'. Nuevamente, puede realizar todo esto con una única línea de comandos.

%\item El comando \texttt{cut} selecciona partes del texto en un archivo. Por ejemplo, usando el archivo de datos de Supernovas descrito en la guía 01 (\texttt{SCPUnion2.1\_mu\_vs\_z.txt}, que consta de varias columnas), podemos seleccionar (`cortar') sólo la segunda tercera columna de datos con el comando
%\begin{minted}[bgcolor=bg]{bash}
%cut -s -f 3 SCPUnion2.1_mu_vs_z.txt
%\end{minted}
%\begin{parts}
%\item Verifique que la opción \texttt{-s} sirve para que \texttt{cut} se salte las primeras líneas del archivo, que contienen comentarios y no datos.
%\item Verifique que la opción \texttt{-f 3} selecciona la tercera columna (pruebe seleccionando otras columnas).
%\end{parts}

%\item Use el editor de texto \texttt{nano} en la consola y cree un archivo de texto llamado \texttt{plantilla.tex}, con el siguiente contenido (que usará en la primera práctica de \LaTeX, la próxima semana):
%
%\begin{minted}[bgcolor=bg]{tex}
%\documentclass[12pt]{article}
%
%\begin{document}
%\'Este es mi primer documento en \LaTeX
% 
%\end{document}
%\end{minted}
\end{questions}
\end{document} 