\documentclass[11pt]{exam}
\usepackage[activeacute,spanish]{babel} % Permite el idioma espa\~nol.
\usepackage[utf8]{inputenc}
\usepackage{amsmath,epsfig}
\usepackage[colorlinks]{hyperref}

\usepackage{minted} 
\usemintedstyle{emacs}
\usepackage{tcolorbox} % colores para el fondo
\definecolor{bg}{rgb}{0.95,0.95,0.95} %color de fondo

\pagestyle{headandfoot}
\spanishdecimal{.}

\begin{document}

\firstpageheadrule
\runningheadrule
\lhead{\bf\normalsize Computaci\'on Cient\'ifica\\ Gu\'ia 11}
\rhead{\bf\normalsize Cs. F\'is., Astro., Geof\'is. \\ 2021-1}
\chead{\bf\normalsize Depto. de F\'isica \\ Universidad de Concepci\'on}
\cfoot{ }
\lfoot{\tiny GR}
\begin{flushleft}
\vspace{0.2in}

\vspace{0.25cm}
\end{flushleft}
%%%%%%%%%%%%%%%%%%%%%%%%%%%%%%%%%%%%%%%%%%

\begin{questions}

%\item Una part'icula realiza un movimiento vertical bajo la influencia de la gravedad de modo que su altura $z(t)$ respecto al suelo es dada por la siguiente ecuaci'on de la trayectoria,
%\begin{equation}
%z(t)=z_0 + v_0 t -\frac{1}{2}gt^2,
%\end{equation}
%con $g=9.8 {\ \rm m/s^2}$. Considere el caso en que $z_0=1 {\ \rm m}$ y $v_0=24 {\ \rm m/s}$.
%
%Escriba un programa en Python que, usando un ciclo \texttt{for}, calcule e imprima el valor de la altura $z(t)$ para los siguientes valores de tiempo (en segundos): $t=0, 0.1, 0.2, \dots 5.0$ (51 valores distintos de tiempo).
%
%\item Modifique el programa anterior, para que ahora 'este pregunte al usuario los valores de $z_0$ y $v_0$. Para esto, use el comando \texttt{input} que aprendi'o en su trabajo con la gu'ia 10.
%
%\item Escriba un programa en Python (que use un ciclo \texttt{for}) que calcule e imprima la suma de los primeros 1000 n'umeros enteros, es decir, el valor de 
%\begin{equation}
%1 + 2 + 3 + 4  + \cdots + 999 + 1000.
%\end{equation}

%\item El factorial de un n'umero entero positivo $n$, denotado por $n!$ es definido por
%\begin{equation}
%n!=1\cdot 2\cdot 3\cdots (n-1)\cdot n.
%\end{equation}
%Por ejemplo, $3!=1\cdot 2\cdot 3=6$ y $10!=3628800$.
%Escriba un programa en Python que pregunte al usuario por el valor de $n$, y que calcule e imprima su factorial, es decir, $n!$.

%\item Realice, ahora escribiendo programas en Python, las mismas tareas planteadas en los ejercicios 3, 4, 8 y 9 de la guía 06, y además los ejercicios 4 al 9 de la sección ``Repetitivas"\, de la guía 07. \textbf{Indicación}: intente usar las \textbf{listas} de Python en tantos lugares como pueda, esto debiese simplificar sus programas y hacerlos más cortos y eficientes.
%
%\item En el texto de referencia de Python usado en clases, \url{https://github.com/gfrubi/CC/blob/master/Python/01-Introduccion-a-la-Programacion-en-Python.ipynb}, lea la secci'on \textbf{Control de Flujo}. En particular, aprenda sobre los comandos \texttt{if}, \texttt{else}, \texttt{elif}.  Reproduzca todos los ejemplos all'i descritos. Puede resultarle útil consultar \href{https://udec.instructure.com/courses/684/files/folder/Python?preview=557597}{estas láminas} (Canvas) con ejemplos y diagramas de flujo.

\item Estudie los distintos casos de uso de los comandos \texttt{if}, \texttt{elif} y \texttt{else} descritos en \href{https://udec.instructure.com/courses/17852/files/folder/Python?preview=1040768}{este pdf} disponible en Canvas. Realice una modificación del código de cada caso, agrengando una función \texttt{input} para que se pueda ingresar el valor de \texttt{x} desde el teclado, y asegúrese que entiende la logica de cómo funciona su código.

\item Escriba un programa que al ejecutarlo pregunte al usuario un n'umero e imprima su valor absoluto. Recuerde que el valor absoluto (o m'odulo) $|x|$ de un valor real $x$ es definido por
\begin{equation}
|x|:=\left\{\begin{array}{cl}
x, &\text{si } x>0 \\
-x, & \text{si } x<0 \\
\end{array}\right. .
\end{equation}

\item Usando lo que aprendi'o sobre el comando \texttt{if} y asociados, modifique el programa \texttt{test.py} que cre'o en la gu'ia 10 y que resuelve la ecuaci'on cuadr'atica $ax^2+bx+c=0$, para que ahora el programa informe que existen dos soluciones reales, y las imprima, si el discriminante $b^2-4ac$ es positivo, o que informe que no existe soluci'on real (si el discriminante es negativo), o bien que informe que existe s'olo una soluci'on real, y la imprima (si el discriminante es nulo).

%\item Modifique ahora el c'odigo que cre'o para calcular el factorial de un n'umero entero para que ahora su programa verifique, antes de calcular el factorial, que el n'umero suministrado es realmente un entero positivo, y s'olo calcule el factorial en ese caso, y que en caso contrario informe al usuario que el n'umero ingresado no es apropiado.

%\item Escribir un programa que pregunte al usuario el valor alg'un n'umero natural e imprima todos los números primos que hay hasta ese número. Por ejemplo, si se ingresa el n'umero 8, el programa debe imprimir los n'umeros 2, 3, 5 y 7.


\item Realice, ahora en Python, lo planteado con los ejercicios $1,2,3,4$ y $6$ de la sección ``Condicional"\, de la \href{https://udec.instructure.com/courses/17852/files/1301010?module_item_id=673055}{guía 08}. \textbf{Indicación}: intente usar las \textbf{listas} de Python.

\item Descargue el libro ``Algoritmos y Programaci'on I: Aprendiendo a programar usando Python como herramienta", de la Facultad de Ingenier'ia de la Universidad de Buenos Aires, desde el sitio \url{https://algoritmos1rw.ddns.net/} (secci'on ``Material", archivo ``Apunte"). Link directo \href{https://drive.google.com/file/d/0B0KKEIBDHL7tdEQ3bFZ2M3VrZzA/view}{aqu\'i}. Atesórelo y estúdielo el resto de sus días.
\end{questions}
\end{document} 