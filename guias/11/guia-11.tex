\documentclass[11pt]{exam}
\usepackage[spanish]{babel} % Permite el idioma español.
\usepackage[utf8]{inputenc}
\usepackage{amsmath}
\usepackage[colorlinks]{hyperref}

\usepackage{minted} 
\usemintedstyle{emacs}
\usepackage{tcolorbox} % colores para el fondo
\definecolor{bg}{rgb}{0.95,0.95,0.95} %color de fondo

\pagestyle{headandfoot}
\spanishdecimal{.}

\begin{document}

\firstpageheadrule
%\firstpagefootrule
%\firstpagefooter{}{Pagina \thepage\ de \pages}{}
\runningheadrule
%\runningfootrule
\lhead{\bf\normalsize Computación Científica\\ Guía 11}
\rhead{\bf\normalsize Cs. Fís., Astro., Geofís. \\ 2024-1}
\chead{\bf\normalsize Depto. de Física \\ Universidad de Concepción}
%\rfoot{\thepage\ / pages}
\cfoot{ }
\lfoot{\tiny GR}
\begin{flushleft}
\vspace{0.2in}
%\hbox to \textwidth{Nombre: \enspace \hrulefill}
%Nombre : \\
\vspace{0.25cm}
\end{flushleft}
%%%%%%%%%%%%%%%%%%%%%%%%%%%%%%%%%%%%%%%%%%

\begin{questions}
\item Vea el video en el que se explican los aspectos básicos del uso de módulos en Python, disponible \href{https://udec.instructure.com/courses/51022/pages/modulos?module_item_id=1904638}{aquí}.

\item Copie los códigos que escribió anteriormente y que definen su implementación de la función \texttt{mifactorial(n)}, y además el código que calcula su aproximación de $\pi$ (almacenada ahora en la variable \texttt{mipi}) en un nuevo archivo llamado \texttt{mimodulo.py}. Este archivo puede usarse para definir un nuevo módulo. A continuación, en una sesión interactiva de Python, importe su función usando primero \texttt{import mimodulo} y llame a las funciones que están ahí definidas. 

\item El factorial es una función comúnmente usada, y ya está implementada en diversos módulos populares de Python, por ejemplo, en el módulo \texttt{math}. Para verificar esto, importe el módulo \texttt{math} y verifique que la función \texttt{math.factorial} entrega los mismos valores ya calculados por usted. Lo mismo ocurre con el valor del número $\pi$ (\texttt{math.pi}). 

\item Aproveche que tiene cargado el módulo \texttt{math} e investigue qué funciones y variables están definidas en este módulo. Para esto, en una sesión interactiva de Python ejecute \texttt{dir(math)}, o bien el comando \texttt{help(math)} para revisar qué contiene. Alternativamente, o revise la \href{https://docs.python.org/3/library/math.html}{documentaci\'on en l\'inea} disponible.

\item Una instalación típica de Python incluye los módulos de la ``\textit{librería estandar}'', con diversas herramientas para realizar una gran variedad de tareas en Python. Ver por ejemplo, \href{https://docs.python.org/es/3/library/index.html}{esta página} para breve introducción.


\item En el módulo \texttt{glob} de la librería estándar está implementada la función \texttt{glob.glob}, que crea una \textit{lista} de los strings de los nombres de los archivos y/o carpetas disponibles en la carpeta del computador en la que se está ejecutando un programa. Para verificar esto, ejecute

\begin{minted}[bgcolor=bg]{python}
import glob

lista_todos = glob.glob('*')
print('Lista de todos los archivos y carpetas en la carpeta actual:')
print(lista_todos)

lista_py = glob.glob('*.py')
print('Lista de todos los archivos .py en la carpeta actual:')
print(lista_py)
print('Numero total de archivos .py en la carpeta actual = ', len(lista_py))
\end{minted}

\item En el módulo \texttt{os} puede encontrar herramientas para manipular archivos. Por ejemplo \\ \texttt{os.rename(antiguo,nuevo)} renombra el archivo cuyo nombre original está dado por el string \texttt{antiguo} a un nuevo nombre correspondiente al string \texttt{nuevo}. Así, si en la carpeta donde se está ejecutando su programa existe un archivo llamado ``\texttt{01.py}'', entonces el comando 
\begin{minted}[bgcolor=bg]{python}
os.rename('01.py','01-old.py')
\end{minted}

lo renombrará a ``\texttt{01-old.py}''

\item Note que el módulo \texttt{os} también implementa una función similar a \texttt{glob.glob}, llamada \texttt{os.listdir}. Para comprobarlo, ejecute
\begin{minted}[bgcolor=bg]{python}
os.listdir()
\end{minted}

\item Usando lo anterior, escriba un programa en Python que renombre todos los archivos de extensión \texttt{.txt} en una carpeta, asignándoles un nuevo nombre con un número correlativo (es decir `\texttt{01.txt}', `\texttt{02.txt}', `\texttt{03.txt}', etc.).


\item Revise los links a módulos de interés en Física, Geofísica y Astronomía listados en el documento disponible \href{https://github.com/gfrubi/CC/blob/master/Python/01-Introduccion-a-la-Programacion-en-Python.ipynb}{aquí} (busque la parte donde dice ``Aquí listamos algunos módulos generales útiles ..."). Intente familiarizarse con lo que hace cada uno de estos módulos.
\end{questions}

\end{document} 