\documentclass[11pt]{exam}
\usepackage[activeacute,spanish]{babel} % Permite el idioma espa\~nol.
\usepackage[utf8]{inputenc}
\usepackage{amsmath}
\usepackage[colorlinks]{hyperref}

\pagestyle{headandfoot}

\begin{document}

\firstpageheadrule
%\firstpagefootrule
%\firstpagefooter{}{Pagina \thepage\ de \pages}{}
\runningheadrule
%\runningfootrule
\lhead{\bf\normalsize Computación Científica\\ Guía 05}
\rhead{\bf\normalsize Cs. Fís., Astro., Geofís. \\ 2023-1}
\chead{\bf\normalsize Depto. de Física \\ Universidad de Concepción}
%\rfoot{\thepage\ / pages}
\cfoot{ }
\lfoot{\tiny GR}
\begin{flushleft}
\vspace{0.2in}
%\hbox to \textwidth{Nombre: \enspace \hrulefill}
%Nombre : \\
\vspace{0.25cm}
\end{flushleft}
%%%%%%%%%%%%%%%%%%%%%%%%%%%%%%%%%%%%%%%%%%

\begin{questions}
\item Conozca uno de los editores de texto especializados en edición de código \LaTeX. La persona encargada de la práctica les mostrará las funcionalidades básicas del programa \href{http://www.xm1math.net/texmaker/}{TexMaker} (multiplataforma, existen versiones para Linux, Windows y MacOS) o bien de \href{https://kile.sourceforge.io/}{Kile} (versiones para Linux y Windows). Estos programas son muy similares\footnote{De hecho, TexMaker es la ``continuación'' del proyecto Kile, por el mismo autor (Pascal Brachet) que escribió originalemente este último software.} y proporcionan muchas características que hacen de la edición de código \LaTeX\, una labor más rápida y cómoda. En esta semana, intente usar exclusivamente uno de estos programas.

\item Lea el contenido de las páginas 31 a 36 del \href{https://udec.instructure.com/courses/17852/pages/latex-pdf-presentacion?module_item_id=531012}{pdf de clases}. Reproduzca los ejemplos ahí descritos. Si tiene dudas, consulte con su ayudante.

\item Entre los temas discutidos en el punto anterior está el entorno \texttt{equation}. Este entorno define una línea especial para ingresar expresiones en modo matemático, de modo que éstas aparecen centradas en la página, con un tamaño un poco más grande, y con una numeración automática a la derecha. Reproduzca los ejemplos de la página 34 y 35, es decir, escriba el código indicado en un archivo \texttt{.tex} y verifique que realiza lo señalado.

\item Una de las preguntas de un certamen de a\~nos anteriores consistió en pedirle a l@s estudiantes que escribieran (con lápiz y papel) el código \LaTeX\ que reproduce la famosa ecuación de \href{https://es.wikipedia.org/wiki/Erwin_Schr\%C3\%B6dinger}{Schr\"odinger} (una de la ecuaciones fundamentales de la Mecánica Cuántica):
\begin{equation}
-\frac{\hbar^2}{2m}\nabla^2\Psi+V(\vec{x})\Psi=i\hbar\frac{\partial\Psi}{\partial t}.
\end{equation}
Escriba el código \LaTeX\ que reproduce esta hermosa ecuación.

\item Ahora que sabe cómo escribir la ecuación de Schr\"odinger en \LaTeX, mire el artículo de Wikipedia correspondiente, \href{https://es.wikipedia.org/wiki/Ecuaci\%C3\%B3n_de_Schr\%C3\%B6dinger}{aquí}. Encuentre alguna de las versiones allí escritas de esta ecuación e ingrese al link marcado como [\texttt{editar}] más cercano. Observe ahí cómo luce el código que genera el artículo en cuestión. Como puede ver, las ecuaciones en Wikipedia se generan con un código que es esencialmente \LaTeX, acompa\~nado con algunos otros códigos especiales. Entonces, a partir de ahora usted sabe cómo editar/agregar ecuaciones en artículos de Wikipedia!

\item Siga entrenando sus poderes en lenguaje \LaTeX, reproduciendo estas expresiones:
\begin{equation}
\int \sin^2(x)\,dx=\frac{x-\sin (x)\,\cos (x)}{2}  ,
\end{equation}
\begin{equation}
v = c \ \sqrt{1- \frac{{m^2 c^4}}{{(mc^2+K)^2}}},
\end{equation}
\begin{equation}
\vec{F} = \frac{\text{d}\vec{p}}{\text{d}t} = \frac{\text{d}(\gamma m \vec{v})}{\text{d}t} = m \gamma^3 \vec{a} = \frac{m \vec{a}}{[1-(v^2/c^2)]^{3/2}}.
\end{equation}


\item En \href{https://github.com/gfrubi/CC/blob/master/guias/05/ejemplo-g5.pdf}{este archivo pdf} encontrará el extracto de un texto simple de Física (Mecánica) que hace uso de ecuaciones (entorno \texttt{equation}), fórmulas en línea, (sub)secciones, y diversos símbolos matemáticos. Escriba un código \LaTeX\ que reproduzca lo más fielmente posible el contenido de este pdf.
\end{questions}
\end{document} 