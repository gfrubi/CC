\documentclass[11pt]{exam}
\usepackage[activeacute,spanish]{babel} % Permite el idioma espa\~nol.
\usepackage[utf8]{inputenc}
\usepackage{amsmath}
\usepackage[colorlinks]{hyperref}

\usepackage{minted} 
\usemintedstyle{emacs}
\usepackage{tcolorbox} % colores para el fondo
\definecolor{bg}{rgb}{0.95,0.95,0.95} %color de fondo

\pagestyle{headandfoot}

\begin{document}

\firstpageheadrule
%\firstpagefootrule
%\firstpagefooter{}{Pagina \thepage\ de \pages}{}
\runningheadrule
%\runningfootrule
\lhead{\bf\normalsize Computaci\'on Cient\'ifica\\ Gu\'ia 05}
\rhead{\bf\normalsize Cs. F\'is., Astro., Geof\'is. \\ 2021-1}
\chead{\bf\normalsize Depto. de F\'isica \\ Universidad de Concepci\'on}
%\rfoot{\thepage\ / pages}
\cfoot{ }
\lfoot{\tiny GR}
\begin{flushleft}
\vspace{0.2in}
%\hbox to \textwidth{Nombre: \enspace \hrulefill}
%Nombre : \\
\vspace{0.25cm}
\end{flushleft}
%%%%%%%%%%%%%%%%%%%%%%%%%%%%%%%%%%%%%%%%%%

%\begin{center}
%\texttt{Fecha de Entrega: Jueves 28 de Agosto. Env'ie a gfrubi@udec.cl los archivos .py que resuelven cada uno de los problemas propuestos.}
%\end{center}
\begin{questions}

\item Cree un nuevo documento \texttt{.tex} que cargue los paquetes \texttt{amsmath} y \texttt{amsfonts} 

(use \verb|\usepackage{amsmath,amsfonts}|) y que genere las siguientes expresiones matem\'aticas:
\begin{itemize}

\item La siguiente ecuaci'on numerada:
\begin{equation}\label{asinNnu}
Y_\nu(x)\approx \sqrt{\frac{2}{\pi x}}\sen\left(x-\frac{\nu\pi}{2}-\frac{\pi}{4}\right), 
\qquad x\gg\left|\nu^2-\frac{1}{4}\right|.
\end{equation}
Usando 

\begin{minted}[bgcolor=bg]{tex}
	\begin{equation}

	\end{equation}
\end{minted}
\item La siguiente ecuaci'on no numerada:
\begin{equation*}
\int_{0}^{\infty }\frac{\log (x)}{x^{2}}dx=-\left. \frac{\log (x)}{x}\right]
_{0}^{\infty }+\int_{0}^{\infty }\frac{1}{x^{2}}dx
\end{equation*}
Usando

\begin{minted}[bgcolor=bg]{tex}
	\begin{equation*}

	\end{equation*}
\end{minted}
\item La siguiente expresi'on del m'ultiples l'ineas:
\begin{align} 
(a+b)^4 &= (a+b)^2 (a+b)^2 \\
 &= (a^2+2ab+b^2) (a^2+2ab+b^2) \\
 &= a^4+4 a^3 b + 6 a^2 b^2 +4 a b^3 +b^4
\end{align}
Usando	

\begin{minted}[bgcolor=bg]{tex}

	\begin{align} 
	 ... &= ...\\
	 ... &= ...\\
	 ... &= ...
	\end{align} 
\end{minted}

\item Una expresi'on enmarcada, usando el comando \verb|\boxed{}| de \texttt{amsmath}:
\newline
\begin{equation*}
\boxed{\int u \, dv=u\,v-\int v \, du}
\end{equation*}

\item Diferentes tipografías matem\'aticas con fuentes de \texttt{amsfonts}:
\begin{itemize}
\item 
\begin{equation}
\mathsf{C}_{ijkl}=\boldsymbol{\mathsf{C}}
\end{equation}
Usando

\begin{minted}[bgcolor=bg]{tex}
	\mathsf{C}_{ijkl}=\boldsymbol{\mathsf{C}}.
\end{minted}

\item 
\begin{equation}
\mathcal{A} \neq \boldsymbol{\mathcal{A}}
\end{equation}
Usando

\begin{minted}[bgcolor=bg]{tex}
	\mathcal{A} \neq \boldsymbol{\mathcal{A}}
\end{minted}


\end{itemize}
\end{itemize}
\item La siguiente expresi\'on
\begin{equation}
 |x| = \left\{ \begin{array}{ll}
         x & \mbox{si $x \geq 0$},\\
        -x & \mbox{si $x < 0$}.\end{array} \right.
\end{equation}

\item Siga entrenando sus poderes en lenguaje \LaTeX, reproduciento estas expresiones:
\begin{equation}
\Lambda^{\mu'}{}_\nu = \begin{pmatrix}
\gamma & -\beta\gamma/c & 0 & 0\\
-\beta\gamma c & \gamma & 0 & 0\\
0 & 0 & 1 & 0\\
0 & 0 & 0 & 1 \end{pmatrix},
\end{equation}
\begin{equation}
U^\mu = \frac{\text{d}x^\mu}{\text{d}\tau} =
\begin{pmatrix} \gamma \\ \gamma v_x \\ \gamma v_y \\ \gamma v_z \end{pmatrix}.
\end{equation}

\item En su archivo de trabajo, agregue algunas referencias bibliogr'aficas, como por ejemplo las que aparecen al final de este archivo (ojo con los tipos de letras!).

\item Agregue el paquete de idiomas \texttt{babel}, usando \verb|\usepackage[spanish]{babel}| y vea c'omo afecta al resultado obtenido.

\item Agregue ahora el paquete \texttt{hyperref} agregando 
\verb|\usepackage[colorlinks]{hyperref}| a su archivo (antes del comando \verb|\begin{document}|). Este paquete agrega autom'aticamente \textit{hyperlinks} a su pdf. Verique que ahora puede hacer click en los n'umeros de ecuaciones citados, as'i como en los n'umeros correspondientes a las referencias bibliogr'aficas. Note que \texttt{hyperref} tambi'en agrega hyperlinks a la tabla de contenidos de su pdf, si 'esta existe.

\item El paquete \texttt{hyperref} tambi'en permite incluir \textit{hyperlinks externos} a su pdf. Existen b'asicamente dos formas: La m'as simple es incluir comandos como \verb|\url{http://www.cfm.cl}|, que agrega un hyperlink a la direcci'on se\~nalada. Agregue un ejemplo de este tipo a su archivo de trabajo y vea el resultado.

\item La segunda forma de agregar hyperlinks usando \texttt{hyperref} es con un comando de la forma \verb|\url{link}{texto}|, por ejemplo \verb|\href{http://www.cfm.cl}{FCFM}|, que genera un link a la misma direcci'on anterior, pero que ahora aparece bajo el texto ``FCFM'' en el pdf. Incluya un ejemplo de esto en su archivo de trabajo.

\end{questions}
\end{document} 