\documentclass[11pt]{exam}
\usepackage[activeacute,spanish]{babel} % Permite el idioma espa\~nol.
\usepackage[utf8]{inputenc}
\usepackage{amsmath,epsfig}
\usepackage[colorlinks]{hyperref}

\usepackage{minted} 
\usemintedstyle{emacs}
\usepackage{tcolorbox} % colores para el fondo
\definecolor{bg}{rgb}{0.95,0.95,0.95} %color de fondo

\pagestyle{headandfoot}

\begin{document}

\firstpageheadrule
%\firstpagefootrule
%\firstpagefooter{}{Pagina \thepage\ de \pages}{}
\runningheadrule
%\runningfootrule
\lhead{\bf\normalsize Computaci\'on Cient\'ifica\\ Gu\'ia 06}
\rhead{\bf\normalsize Cs. F\'is., Astro., Geof\'is. \\ 2021-1}
\chead{\bf\normalsize Depto. de F\'isica \\ Universidad de Concepci\'on}
%\rfoot{\thepage\ / pages}
\cfoot{ }
\lfoot{\tiny GR}
\begin{flushleft}
\vspace{0.2in}
%\hbox to \textwidth{Nombre: \enspace \hrulefill}
%Nombre : \\
\vspace{0.25cm}
\end{flushleft}
%%%%%%%%%%%%%%%%%%%%%%%%%%%%%%%%%%%%%%%%%%

\begin{questions}
%\item Vea el video resumen sobre el uso de referencias cruzadas en \LaTeX, disponible en Canvas \href{https://udec.instructure.com/courses/17852/pages/referencias-cruzadas?module_item_id=667756}{aquí}.
%\item Vea el video resumen sobre el uso de referencias bibliográficas en \LaTeX, disponible en Canvas \href{https://udec.instructure.com/courses/17852/pages/referencias-bibliograficas?module_item_id=667757}{aquí}.
%\item Vea el video resumen sobre la creación de tablas en \LaTeX, disponible en Canvas \href{https://udec.instructure.com/courses/17852/pages/tablas?module_item_id=667758}{aquí}. Siga las sugerencias ahí descritas para familiarizarse con las distintas posibilidades disponibles.
\item Vea el video resumen sobre cómo incluir gráficos en un documento \LaTeX, disponible en Canvas \href{https://udec.instructure.com/courses/29314/pages/figuras?module_item_id=1039065}{aquí}.
\item En alg'un archivo .tex que ya tenga hecho, inserte el escudo de la UdeC, contenido en el archivo \url{http://www.udec.cl/normasgraficas/sites/default/files/marcaderecha.png}:
\begin{figure}[h!]
\begin{center}
\includegraphics[width=4cm]{marcaderecha.png}
\end{center}
\caption{Marca alineaci'on derecha, Formato PNG.}
\label{fig:escudo}
\end{figure}

Vea \href{http://www.udec.cl/normasgraficas/node/7}{esta} p'agina para conocer otras variaciones oficiales del escudo de la UdeC. 
Puede encontrar m'as informaci'on acerca de las normas gr'aficas de nuestra Universidad en \href{http://www.udec.cl/normasgraficas}{esta} p'agina.

\item Ahora su misi'on es escribir su primer ``art'iculo cient'ifico'' en \LaTeX, que debe reproducir lo m'as fielmente posible \href{https://github.com/gfrubi/CC/blob/master/guias/06/ejemplo-articulo.pdf}{este} ejemplo. Para eso, use todo lo aprendido (en particular, use referencias cruzadas a las ecuaciones, tablas, figuras y referencias). El archivo .pdf de la figura lo puede descargar desde \href{https://github.com/gfrubi/CC/blob/master/guias/06/fig-ajuste-lineal.pdf}{aqu\'i}.

El t'itulo, autor(a), y resumen puede ser incorporados usando los siguientes comandos despu'es del conocido \verb|\documentclass|:

\begin{minted}[bgcolor=bg]{tex}
\title{Escribiendo mi primer artículo con formato Científico, en \LaTeX}
\author{Su nombre (autor(a))}

\begin{document}

\maketitle
\begin{abstract}
Este es el resumen del artículo...
\end{abstract}
\end{minted}

\item Finalmente, cree una presentación usando la clase \texttt{Beamer}.
\begin{parts}
\item Lea la página 66 del \href{https://udec.instructure.com/courses/29314/pages/latex-pdf-presentacion?module_item_id=1039048}{pdf de clases}.
\item Descargue el archivo de ejemplo \href{https://github.com/gfrubi/CC/blob/master/guias/06/ejemplo-beamer.tex}{ejemplo-beamer.tex} y estudie su contenido.
\item Explore cómo lucen los distintos temas disponibles en Beamer, cambiando la opción \verb|\usetheme{}|.
\item Adapte este ejemplo para crear su propia, única y espectacular presentación. Ésta debe incluir al menos: una ecuación (entorno \texttt{equation}), una ecuación alineada (entorno \texttt{eqnarray} o \texttt{align}), una tabla y una figura.
\item Comparta una copia del archivo pdf su presentación en el \href{https://teams.microsoft.com/l/channel/19\%3a39b89cfa8e2d4deaa611d18190ed6c9d\%40thread.tacv2/02\%2520Pr\%25C3\%25A1cticas?groupId=787ade39-1121-4cac-99ae-cc187b6e6587&tenantId=56582b9e-8824-49d0-a665-cd328c0e004a}{canal de prácticas} del grupo de Teams del curso.
\end{parts}
\end{questions}

\end{document} 