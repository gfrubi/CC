\documentclass[11pt]{exam}
\usepackage[spanish]{babel} % Permite el idioma español.
\usepackage[utf8]{inputenc}
\usepackage{amsmath}
\usepackage[colorlinks]{hyperref}

\usepackage{minted} 
\usemintedstyle{emacs}
\usepackage{tcolorbox} % colores para el fondo
\definecolor{bg}{rgb}{0.95,0.95,0.95} %color de fondo

\pagestyle{headandfoot}

\begin{document}

\firstpageheadrule
%\firstpagefootrule
%\firstpagefooter{}{Pagina \thepage\ de \pages}{}
\runningheadrule
%\runningfootrule
\lhead{\bf\normalsize Computación Científica\\ Guía 06}
\rhead{\bf\normalsize Cs. Fís., Astro., Geofís. \\ 2024-1}
\chead{\bf\normalsize Depto. de Física \\ Universidad de Concepción}
%\rfoot{\thepage\ / pages}
\cfoot{ }
\lfoot{\tiny GR}
\begin{flushleft}
\vspace{0.2in}
%\hbox to \textwidth{Nombre: \enspace \hrulefill}
%Nombre : \\
\vspace{0.25cm}
\end{flushleft}
%%%%%%%%%%%%%%%%%%%%%%%%%%%%%%%%%%%%%%%%%%

%\begin{center}
%\texttt{Fecha de Entrega: Jueves 28 de Agosto. Envíe a gfrubi@udec.cl los archivos .py que resuelven cada uno de los problemas propuestos.}
%\end{center}
\begin{questions}

\item Cree un nuevo documento \texttt{.tex} que cargue los paquetes \texttt{amsmath} y \texttt{amsfonts} 

(use \verb|\usepackage{amsmath,amsfonts}|) y que genere las siguientes expresiones matemáticas:
\begin{itemize}

\item La siguiente ecuación numerada:
\begin{equation}\label{asinNnu}
Y_\nu(x)\approx \sqrt{\frac{2}{\pi x}}\sin\left(x-\frac{\nu\pi}{2}-\frac{\pi}{4}\right), 
\qquad x\gg\left|\nu^2-\frac{1}{4}\right|.
\end{equation}
Usando 

\begin{minted}[bgcolor=bg]{tex}
	\begin{equation}

	\end{equation}
\end{minted}
\item La siguiente ecuación no numerada:
\begin{equation*}
\int_{0}^{\infty }\frac{\log (x)}{x^{2}}dx=-\left. \frac{\log (x)}{x}\right]
_{0}^{\infty }+\int_{0}^{\infty }\frac{1}{x^{2}}dx
\end{equation*}
Usando

\begin{minted}[bgcolor=bg]{tex}
	\begin{equation*}

	\end{equation*}
\end{minted}
\item La siguiente expresión del múltiples líneas:
\begin{align} 
(a+b)^4 &= (a+b)^2 (a+b)^2 \\
 &= (a^2+2ab+b^2) (a^2+2ab+b^2) \\
 &= a^4+4 a^3 b + 6 a^2 b^2 +4 a b^3 +b^4
\end{align}
Usando (ver páginas 47-48 de la \href{https://udec.instructure.com/courses/40179/pages/latex-pdf-presentacion?module_item_id=1465499}{presentación de clases})

\begin{minted}[bgcolor=bg]{tex}

	\begin{align} 
	 ... &= ...\\
	 ... &= ...\\
	 ... &= ...
	\end{align} 
\end{minted}

\item Una expresión enmarcada, usando el comando \verb|\boxed{}| de \texttt{amsmath}:
\newline
\begin{equation*}
\boxed{\int u \, dv=u\,v-\int v \, du}
\end{equation*}

\item Diferentes tipografías matemáticas con fuentes de \texttt{amsfonts}:
\begin{itemize}
\item 
\begin{equation}
\mathsf{C}_{ijkl}=\boldsymbol{\mathsf{C}}
\end{equation}
Usando

\begin{minted}[bgcolor=bg]{tex}
	\mathsf{C}_{ijkl}=\boldsymbol{\mathsf{C}}.
\end{minted}

\item 
\begin{equation}
\mathcal{A} \neq \boldsymbol{\mathcal{A}}
\end{equation}
Usando

\begin{minted}[bgcolor=bg]{tex}
	\mathcal{A} \neq \boldsymbol{\mathcal{A}}
\end{minted}


\end{itemize}
\end{itemize}
\item La siguiente expresión
\begin{equation}
 |x| = \left\{ \begin{array}{ll}
         x & \mbox{si $x \geq 0$},\\
        -x & \mbox{si $x < 0$}.\end{array} \right.
\end{equation}

\item Siga entrenando sus poderes en lenguaje \LaTeX, reproduciendo estas expresiones:
\begin{equation}
\Lambda^{\mu'}{}_\nu = \begin{pmatrix}
\gamma & -\beta\gamma/c & 0 & 0\\
-\beta\gamma c & \gamma & 0 & 0\\
0 & 0 & 1 & 0\\
0 & 0 & 0 & 1 \end{pmatrix},
\end{equation}
\begin{equation}
U^\mu = \frac{\text{d}x^\mu}{\text{d}\tau} =
\begin{pmatrix} \gamma \\ \gamma v_x \\ \gamma v_y \\ \gamma v_z \end{pmatrix}.
\end{equation}

\item En su archivo de trabajo, agregue algunas referencias bibliográficas, siguiendo el ejemplo en la página página 53 de la \href{https://udec.instructure.com/courses/51022/pages/latex-pdf-presentacion?module_item_id=1904611}{presentación de clases}.

\item Agregue el paquete de idiomas \texttt{babel}, usando \verb|\usepackage[spanish]{babel}| y vea cómo afecta al resultado obtenido.

\item Agregue ahora el paquete \texttt{hyperref} agregando 
\verb|\usepackage[colorlinks]{hyperref}| a su archivo (antes del comando \verb|\begin{document}|). Este paquete agrega automáticamente \textit{hyperlinks} a su pdf. Verique que ahora puede hacer click en los números de ecuaciones citados, así como en los números correspondientes a las referencias bibliográficas. Note que \texttt{hyperref} también agrega hyperlinks a la tabla de contenidos de su pdf, si ésta existe.

\item El paquete \texttt{hyperref} también permite incluir \textit{hyperlinks externos} a su pdf. Existen básicamente dos formas: La más simple es incluir comandos como \verb|\url{http://www.cfm.cl}|, que agrega un hyperlink a la dirección señalada. Agregue un ejemplo de este tipo a su archivo de trabajo y vea el resultado.

\item La segunda forma de agregar hyperlinks usando \texttt{hyperref} es con un comando de la forma \verb|\url{link}{texto}|, por ejemplo \verb|\href{http://www.cfm.cl}{FCFM}|, que genera un link a la misma direccion anterior, pero que ahora aparece bajo el texto ``FCFM'' en el pdf. Incluya un ejemplo de esto en su archivo de trabajo.

\item Vea el video resumen sobre la creación de tablas en \LaTeX, disponible en Canvas \href{https://udec.instructure.com/courses/51022/pages/referencias-cruzadas}{aquí}. Siga las sugerencias ahí descritas para familiarizarse con las distintas posibilidades disponibles.

%\item Vea el video resumen sobre cómo incluir figuras/gráficos en un documento \LaTeX, disponible en Canvas \href{https://udec.instructure.com/courses/40179/pages/figuras?module_item_id=1465511}{aquí}.
%\item En algún archivo .tex que ya tenga hecho, inserte el escudo de la UdeC, contenido en el archivo \url{http://www.udec.cl/normasgraficas/sites/default/files/marcaderecha.png}:
%\begin{figure}[h!]
%\begin{center}
%\includegraphics[width=4cm]{marcaderecha.png}
%\end{center}
%\caption{Marca alineación derecha, Formato PNG.}
%\label{fig:escudo}
%\end{figure}
%
%Vea \href{http://www.udec.cl/normasgraficas/node/7}{esta} página para conocer otras variaciones oficiales del escudo de la UdeC. 
%Puede encontrar más información acerca de las normas gráficas de nuestra Universidad en \href{http://www.udec.cl/normasgraficas}{esta} página.
%
%\item Ahora su misión es escribir su primer ``artículo científico'' en \LaTeX, que debe reproducir lo más fielmente posible \href{https://github.com/gfrubi/CC/blob/master/guias/06/ejemplo-articulo.pdf}{este} ejemplo. Para eso, use todo lo aprendido (en particular, use referencias cruzadas a las ecuaciones, tablas, figuras y referencias). El archivo .pdf de la figura lo puede descargar desde \href{https://github.com/gfrubi/CC/blob/master/guias/06/fig-ajuste-lineal.pdf}{aquí}.
%
%El título, autor(a), y resumen puede ser incorporados usando los siguientes comandos después del conocido \verb|\documentclass|:
%
%\begin{minted}[bgcolor=bg]{tex}
%\title{Escribiendo mi primer artículo con formato Científico, en \LaTeX}
%\author{Su nombre (autor(a))}
%
%\begin{document}
%
%\maketitle
%\begin{abstract}
%Este es el resumen del artículo...
%\end{abstract}
%\end{minted}
\end{questions}
\end{document} 