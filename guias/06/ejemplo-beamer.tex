\documentclass[hyperref={colorlinks},aspectratio=169]{beamer}

\usepackage{graphicx}
%\usepackage{amsfonts,amsmath,color}       
\usepackage[spanish,activeacute]{babel}
\usepackage[utf8]{inputenc}

\usefonttheme{professionalfonts}
\usetheme{Madrid}
% Bergen, Boadilla, Copenhagen, Dresden, Hannover, Luebeck, AnnArbor, Berkeley, Darmstadt, Frankfurt, Ilmenau, Madrid, Warsaw, Antibes, Berlin, CambridgeUS, Malmoe, PaloAlto
%\setBeamercovered{transparent}

\begin{document}

\title{Ejemplo de una presentación usando Beamer}
\author[iniciales de autor(a)]{Autor(a)} 
\frame{\titlepage}

\begin{frame}
\frametitle{Contenidos}
\tableofcontents
\end{frame}

%%%%%%%%%%%%%%%%%%%%%%%%%%%

\section{Introducción}
\begin{frame}[fragile]\frametitle{Título de la lámina}
Podemos incluir un ``bloque'', usando el entorno \texttt{block}
\begin{block}{nombre del bloque}
Una ecuación dentro del bloque
\begin{equation}
\nabla^2\Psi=0
\end{equation}
\end{block}
\end{frame}

\end{document}
