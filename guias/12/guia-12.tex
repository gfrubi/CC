\documentclass[11pt]{exam}
\usepackage[spanish]{babel} % Permite el idioma español.
\usepackage[utf8]{inputenc}
\usepackage{amsmath}
\usepackage[colorlinks]{hyperref}

\usepackage{minted} 
\usemintedstyle{emacs}
\usepackage{tcolorbox} % colores para el fondo
\definecolor{bg}{rgb}{0.95,0.95,0.95} %color de fondo

\pagestyle{headandfoot}
\spanishdecimal{.}

\begin{document}

\firstpageheadrule
%\firstpagefootrule
%\firstpagefooter{}{Pagina \thepage\ de \pages}{}
\runningheadrule
%\runningfootrule
\lhead{\bf\normalsize Computación Científica\\ Guía 12}
\rhead{\bf\normalsize Cs. Fís., Astro., Geofís. \\ 2024-1}
\chead{\bf\normalsize Depto. de Física \\ Universidad de Concepción}
%\rfoot{\thepage\ / pages}
\cfoot{ }
\lfoot{\tiny GR}
\begin{flushleft}
\vspace{0.2in}
%\hbox to \textwidth{Nombre: \enspace \hrulefill}
%Nombre : \\
\vspace{0.25cm}
\end{flushleft}
%%%%%%%%%%%%%%%%%%%%%%%%%%%%%%%%%%%%%%%%%%

\begin{questions}
\item En esta práctica usted ejercitará y explorará algunas características del poderoso módulo \texttt{Numpy}. Para esto, cargue el módulo usando:

\begin{minted}[bgcolor=bg]{python}
import numpy as np
\end{minted}

y (si no asistió a la clase de esta semana) vea el primero de los videos introductorios disponibles en Canvas \href{https://udec.instructure.com/courses/51022/pages/numpy}{aquí}.

\item En \texttt{Numpy} existe una función llamada \texttt{arange} que es muy similar a la ya conocida función \texttt{range}. La diferencia es que \texttt{range} sirve para generar una \textit{lista} mientras que \texttt{arange} genera un \textit{arreglo de \texttt{Numpy}}. Para comprobar esto, ejecute:

\begin{minted}[bgcolor=bg]{python}
x = range(10)
y = np.arange(10)
print(x,type(x))
print(y,type(y))
\end{minted}

En otras palabras, \texttt{np.arange(10)} es equivalente a \texttt{np.array(range(10))}.

\item Usando arreglos de \texttt{Numpy} es posible realizar muchos cálculos en forma rápida y eficiente, sin necesidad de recurrir a ciclos (\texttt{for} o \texttt{while}). Por ejemplo, puede calcular la misma suma considerada en el problema 3 de la guía 11, es decir, 
\begin{equation}
1 + 2 + 3 + 4  + \cdots + 999 + 1000,
\end{equation}
pero ahora usando las función \texttt{sum} de \texttt{Numpy} (que suma todos los elementos de un arreglo):

\begin{minted}[bgcolor=bg]{python}
n = np.arange(1001)
suma = np.sum(n)
print(suma)
\end{minted}
o, en una sola línea

\begin{minted}[bgcolor=bg]{python}
print(np.sum(np.arange(1001)))
\end{minted}
Verifique lo anterior y asegúrese de entender qué se está calculando.

\item Adapte la idea del cálculo en el punto anterior para implementar un cálculo alternativo para el factorial de un número $n$ (entero positivo), pero esta vez usando un arreglo de \texttt{Numpy} y la función \texttt{prod()} que calcula el producto de cada componente de un arreglo de \texttt{Numpy} (similarmente a como \texttt{sum()} calcula la suma).

\item Usando Numpy, calcule el valor de la suma de los primeros 101 términos de la forma
\begin{equation}
1+\left(\frac{1}{2}\right)+\left(\frac{1}{2}\right)^2+\left(\frac{1}{2}\right)^3+\dots +\left(\frac{1}{2}\right)^{100}.
\end{equation}
\item Verifique que, a diferencia de su pariente \texttt{range()}, la función \texttt{arange()} también funciona con pasos decimales, por ejemplo

\begin{minted}[bgcolor=bg]{python}
print(np.arange(1,10,0.3))
\end{minted}

\item Otra función muy útil para crear arreglos de valores en un intervalo es \texttt{linspace()}, que tiene el formato \texttt{linspace(desde,hasta,numerodeelementos)}. Por ejemplo, ejecute los siguientes comandos:

\begin{minted}[bgcolor=bg]{python}
x = np.linspace(1,10,20)
y = np.linspace(-np.pi,np.pi,100)
print(x,np.size(x))
print(y,np.size(y))
\end{minted}

\item Otra propiedad importante de los arreglos es que sus elementos pueden usarse para iterar en un ciclo \texttt{for}. Para ver esto, ejecute:

\begin{minted}[bgcolor=bg]{python}
x = np.arange(11)
y = x**2
for i in x:
	print ("la componente "+str(i)+" de y es igual a "+str(y[i]))
\end{minted}

\item Vea el segundo video disponible en Canvas \href{https://udec.instructure.com/courses/51022/pages/numpy}{aquí}, en el que se explican los aspectos básicos de los arreglos bidimensionales en Numpy.
 
\item Lea sobre el comandos \texttt{shape} y \texttt{len} y \texttt{size} y sobre el\textit{ indexado de arreglos} (tanto uni- como bi-dimensionales) en el \href{https://github.com/gfrubi/CC/blob/master/Python/02-Numpy.ipynb}{archivo sobre Numpy} en el repositorio. Asegúrese de entender los ejemplos ahí discutidos.

\item Descargue el archivo de datos \href{https://udec.instructure.com/courses/51022/pages/numpy}{datos.txt} y guárdelo en la carpeta donde está trabajando. El módulo \texttt{Numpy} contiene una función llamada \texttt{genfromtxt}, que lee datos desde un archivo y los asigna a un arreglo, de la dimensión apropiada. Ejecute (en la misma carpeta donde está el archivo \texttt{datos.txt}) los siguientes comandos:

\begin{minted}[bgcolor=bg]{python}
d = np.genfromtxt("datos.txt")
x = d[:,0]
y = d[:,1]
\end{minted}

La primera línea carga los datos al arreglo \texttt{d}. Las últimas dos líneas asignan la primera columna de datos al arreglo \texttt{x} y la segunda columna a \texttt{y}. Usando las funciones \texttt{shape} y \texttt{size} de \texttt{Numpy}, verifique la forma y tamaño de los arreglos \texttt{d, x} e \texttt{y}. Asegúrese de entender qué es lo que realiza exactamente cada comando anterior.


\item Usando lo anterior, calcule e imprima:
\begin{parts}
\item El promedio de los valores de la primera columna. (puede usar la función \texttt{sum} y \texttt{len} para calcular el promedio, o bien la función \texttt{mean} de \texttt{Numpy}).
\item El promedio \textit{de los cuadrados} de los valores de la segunda columna.
\item La suma de los productos de cada elemento de la primera con la segunda columna (es decir, $0.1*0.738 + 0.25 *	0.826 + 0.41 * 0.981 +\cdots$).
\end{parts}
\end{questions}
\end{document} 