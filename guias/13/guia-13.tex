\documentclass[11pt]{exam}
\usepackage[spanish]{babel} % Permite el idioma espa\~nol.
\usepackage[utf8]{inputenc}
\usepackage{amsmath,epsfig}
\usepackage[colorlinks]{hyperref}

\usepackage{minted} 
\usemintedstyle{emacs}
\usepackage{tcolorbox} % colores para el fondo
\definecolor{bg}{rgb}{0.95,0.95,0.95} %color de fondo

\pagestyle{headandfoot}
\spanishdecimal{.}

\begin{document}

\firstpageheadrule
\runningheadrule
\lhead{\bf\normalsize Computación Científica\\ Guía 13}
\rhead{\bf\normalsize Cs. Fís., Astro., Geofís. \\ 2024-1}
\chead{\bf\normalsize Depto. de Física \\ Universidad de Concepción}
\cfoot{ }
\lfoot{\tiny GR}
\begin{flushleft}
\vspace{0.2in}

\vspace{0.25cm}
\end{flushleft}
%%%%%%%%%%%%%%%%%%%%%%%%%%%%%%%%%%%%%%%%%%

\begin{questions}

\item En esta práctica Usted se familiarizará con el poderoso módulo \href{https://matplotlib.org/}{Matplotlib} que permite crear gráficos de variados tipos. Explore los distintos tipos de gráficos que este módulo puede crear visitando la galeria oficial del proyecto, siguiente \href{https://matplotlib.org/stable/gallery/index}{este link}. Seleccione un par de estos ejemplos, ingresando al correspondiente link en la imagen, copie el código Python del ejemplo, incorpórelo a un archivo Python y ejecútelo para que vea qué hace.

\item En \href{https://udec.instructure.com/courses/51022/pages/matplotlib?module_item_id=1904643}{este video} disponible en Canvas, se explican los aspectos básicos de Matplotlib. Puede consultarlo si tiene dudas, o bien leer el siguiente \href{https://github.com/PythonUdeC/CPC21/blob/main/04-Matplotlib.ipynb}{este tutorial}. 
%(que es un Jupyter Notebook que puede también descargar).

\item Escriba y ejecute el siguiente programa, que hace uso de \texttt{Numpy} y del módulo gráfico \texttt{Matplotlib}

\begin{minted}[bgcolor=bg]{python}
import matplotlib.pyplot as plt
import numpy as np

d = np.genfromtxt("datos.txt")
x = d[:,0]
y = d[:,1]
plt.plot(x,y, marker="o", markersize=5, color="green", 
	label="Datos experimentales")
plt.title("Voltaje versus Frecuencia")
plt.xlabel("Frecuencia $f$ [Hertz]")
plt.ylabel("Voltaje $V$ [Volt]")
plt.legend()
plt.savefig("g1.pdf")
\end{minted}

Este programa grafica los datos en las listas \texttt{x} e \texttt{y} usando círculos verdes, que guarda en el archivo \texttt{g1.pdf}. El archivo de datos \texttt{datos.txt} puede ser descargado desde \href{https://udec.instructure.com/courses/51022/pages/numpy?module_item_id=1904641}{aquí}.

\item Copie el archivo \texttt{g1.py} a \texttt{g2.py}, que en adelante usará para realizar pruebas. 
\item La opción \texttt{marker="\,o"} indica que los puntos son representados por círculos. Note que, por defecto, estos puntos son unidos por rectas. Otros símbolos (``markers'') disponibles son listados en la tabla \ref{t}. Por ejemplo, la opción \texttt{marker="s"} indica al comando \texttt{plot} que grafique cuadrados. 
Además, la opción \texttt{color} puede adoptar los valores \texttt{blue} (b), \texttt{green} (g), \texttt{red} (r), \texttt{cyan} (c), \texttt{magenta} (m), \texttt{yellow} (y), \texttt{black} (k) y \texttt{white} (w). Puede encontrar más colores listados \href{http://matplotlib.org/examples/color/named_colors.html}{aquí}. Cambie los colores y símbolos del grafico en \texttt{g2.py} para familiarizarse con estas opciones.

\item Agregue una grilla (malla) a su gráfico usando el comando \texttt{plt.grid(True)} antes de \texttt{np.savefig}, y vea qué efecto tiene esto sobre el gráfico creado.
\item Cambie los límites del gráfico agregando los comandos

\begin{minted}[bgcolor=bg]{python}
plt.xlim(0,90)
plt.ylim(0,15)
\end{minted}
y vea el cambio que produce.


\item Exporte el gráfico anterior directamente a formato \texttt{.png}, simplemente cambiando \newline \texttt{plt.savefig("g2.pdf")} por \texttt{plt.savefig("g2.png")} en su programa \texttt{g2.py}. Comparta su lindo gráfico personalizado subiendo el archivo \texttt{p2.png} al foro de Teams de la práctica.

\begin{table}
\begin{center}
\begin{tabular}{cc}
\verb|"."|	& point \\
\verb|","| & pixel \\
\verb|"o"|	& circle \\
\verb|"v"|	& \verb|triangle_down| \\
\verb|"^"|	& \verb|triangle_up| \\
\verb|"<"|	& \verb|triangle_left| \\
\verb|">"|	& \verb|triangle_right| \\
\verb|"1"|	& \verb|tri_down| \\
\verb|"2"|	& \verb|tri_up| \\
\verb|"3"|	& \verb|tri_left| \\
\verb|"4"|	& \verb|tri_right| \\
\verb|"8"|	& octagon \\
\verb|"s"|	& square \\
\verb|"p"|	& pentagon \\
\verb|"*"|	& star \\
\verb|"h"|	& hexagon1 \\
\verb|"H"|	& hexagon2 \\
\verb|"+"|	& plus \\
\verb|"x"|	& x \\
\verb|"D"|	& diamond \\
\verb|"d"|	& \verb|thin_diamond| 
\end{tabular}
\caption{Algunos símbolos disponibles para gráficar puntos con el comando \texttt{plot}. Ver \href{https://matplotlib.org/stable/api/markers_api.html}{este link} para más detalles y símbolos.}
\label{t}
\end{center}
\end{table}

\end{questions}

\end{document} 