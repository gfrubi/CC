\documentclass[11pt]{exam}
\usepackage[activeacute,spanish]{babel} % Permite el idioma espa\~nol.
\usepackage[utf8]{inputenc}
\usepackage{amsmath,epsfig}
\usepackage[colorlinks]{hyperref}

\usepackage{minted} 
\usemintedstyle{emacs}
\usepackage{tcolorbox} % colores para el fondo
\definecolor{bg}{rgb}{0.95,0.95,0.95} %color de fondo

\pagestyle{headandfoot}
\spanishdecimal{.}

\begin{document}

\firstpageheadrule
\runningheadrule
\lhead{\bf\normalsize Computaci\'on Cient\'ifica\\ Gu\'ia 13}
\rhead{\bf\normalsize Cs. F\'is., Astro., Geof\'is. \\ 2022-1}
\chead{\bf\normalsize Depto. de F\'isica \\ Universidad de Concepci\'on}
\cfoot{ }
\lfoot{\tiny GR}
\begin{flushleft}
\vspace{0.2in}

\vspace{0.25cm}
\end{flushleft}
%%%%%%%%%%%%%%%%%%%%%%%%%%%%%%%%%%%%%%%%%%

\begin{questions}


\item En esta pr'actica usted ejercitar'a y explorar'a algunas caracter'isticas del poderoso m'odulo \texttt{Numpy}. Para ello, cargue el m'odulo usando:

\begin{minted}[bgcolor=bg]{python}
import numpy as np
\end{minted}

\item En \texttt{Numpy} existe una funci'on llamada \texttt{arange} que es muy similar a la ya conocida funci'on \texttt{range}. La diferencia es que \texttt{range} genera una \textit{lista} mientras que \texttt{arange} genera un \textit{arreglo de \texttt{Numpy}}. Para comprobar esto, ejecute:

\begin{minted}[bgcolor=bg]{python}
x = range(10)
y = np.arange(10)
print(x,type(x))
print(y,type(y))
\end{minted}

En otras palabras, \texttt{np.arange(10)} es equivalente a \texttt{np.array(range(10))}.

\item Usando arreglos de \texttt{Numpy} es posible realizar muchos c'alculos en forma r'apida y eficiente, sin necesidad de recurrir a ciclos (\texttt{for} o \texttt{while}). Por ejemplo, puede calcular la misma suma considerada en el problema 3 de la guía 11, es decir, 
\begin{equation}
1 + 2 + 3 + 4  + \cdots + 999 + 1000,
\end{equation}
pero ahora usando las función \texttt{sum} de \texttt{Numpy} (que suma todos los elementos de un arreglo):

\begin{minted}[bgcolor=bg]{python}
n = np.arange(1001)
suma = np.sum(n)
print(suma)
\end{minted}
o, en una sola línea

\begin{minted}[bgcolor=bg]{python}
print(np.sum(np.arange(1001)))
\end{minted}
Verifique lo anterior y aseg'urese de entender qu'e se est'a calculando.

\item Adapte la idea del c'alculo en el punto anterior para implementar un c'alculo alternativo para el factorial de un n'umero $n$ (entero positivo), pero esta vez usando un arreglo de \texttt{Numpy} y la funci'on \texttt{prod()} que calcula el producto de cada componente de un arreglo de \texttt{Numpy} (similarmente a como \texttt{sum()} calcula la suma).

\item Usando Numpy, calcule el valor de la suma de los primeros 101 términos de la forma
\begin{equation}
1+\left(\frac{1}{2}\right)+\left(\frac{1}{2}\right)^2+\left(\frac{1}{2}\right)^3+\dots +\left(\frac{1}{2}\right)^{100}.
\end{equation}
\item Verifique que, a diferencia de su pariente \texttt{range()}, la funci'on \texttt{arange()} tambi'en funciona con pasos decimales, por ejemplo

\begin{minted}[bgcolor=bg]{python}
print(np.arange(1,10,0.3))
\end{minted}

\item Otra funci'on muy 'util para crear arreglos de valores en un intervalo es \texttt{linspace()}, que tiene el formato \texttt{linspace(desde,hasta,numerodeelementos)}. Por ejemplo, ejecute los siguientes comandos:

\begin{minted}[bgcolor=bg]{python}
x = np.linspace(1,10,20)
y = np.linspace(-np.pi,np.pi,100)
print(x,np.size(x))
print(y,np.size(y))
\end{minted}

\item Otra propiedad importante de los arreglos es que sus elementos pueden usarse para iterar en un ciclo \texttt{for}. Para ver esto, ejecute:

\begin{minted}[bgcolor=bg]{python}
x = np.arange(11)
y = x**2
for i in x:
	print ("la componente "+str(i)+" de y es igual a "+str(y[i]))
\end{minted}

\item Lea sobre el comandos \texttt{shape} y \texttt{len} y \texttt{size} y sobre el\textit{ indexado de arreglos} (tanto uni- como bi-dimensionales) en el \href{https://github.com/gfrubi/CC/blob/master/Python/02-Numpy.ipynb}{archivo sobre Numpy} en el repositorio. Asegúrese de entender los ejemplos ahí discutidos.

\item Descargue el archivo de datos \href{https://udec.instructure.com/courses/29314/files/folder/guias?preview=2014739}{datos.txt} y guárdelo en la carpeta donde está trabajando. El módulo \texttt{Numpy} contiene una funci'on llamada \texttt{genfromtxt}, que lee datos desde un archivo y los asigna a un arreglo, de la dimensi'on apropiada. Ejecute (en la misma carpeta donde est'a el archivo \texttt{datos.txt}) los siguientes comandos:

\begin{minted}[bgcolor=bg]{python}
d = np.genfromtxt("datos.txt")
x = d[:,0]
y = d[:,1]
\end{minted}

La primera l'inea carga los datos al arreglo \texttt{d}. Las 'ultimas dos l'ineas asignan la primera columna de datos al arreglo \texttt{x} y la segunda columna a \texttt{y}. Usando las funciones \texttt{shape} y \texttt{size} de \texttt{Numpy}, verifique la forma y tama\~no de los arreglos \texttt{d, x} e \texttt{y}. Aseg'urese de entender qué es lo que realiza exactamente cada comando anterior.

\item Usando lo anterior, calcule e imprima:
\begin{parts}
\item El promedio de los valores de la primera columna. (puede usar la funci'on \texttt{sum} y \texttt{len} para calcular el promedio, o bien la funci'on \texttt{mean} de \texttt{Numpy}).
\item El promedio \textit{de los cuadrados} de los valores de la segunda columna.
\item La suma de los productos de cada elemento de la primera con la segunda columna (es decir, $0.1*0.738 + 0.25 *	0.826 + 0.41 * 0.981 +\cdots$).
\end{parts}

\item Escriba y ejecute el siguiente programa, que hace uso de \texttt{Numpy} y del módulo gráfico \texttt{Matplotlib}

\begin{minted}[bgcolor=bg]{python}
import matplotlib.pyplot as plt
import numpy as np

d = np.genfromtxt("datos.txt")
x = d[:,0]
y = d[:,1]
plt.plot(x,y, marker="o", markersize=5, color="green", 
	label="Datos experimentales")
plt.title("Voltaje versus Frecuencia")
plt.xlabel("Frecuencia $f$ [Hertz]")
plt.ylabel("Voltaje $V$ [Volt]")
plt.legend()
plt.savefig("g1.pdf")
\end{minted}

Este programa grafica los datos en las listas \texttt{x} e \texttt{y} usando c'irculos verdes, que guarda en el archivo \texttt{g1.pdf}.
\item Copie el archivo \texttt{g1.py} a \texttt{g2.py}, que en adelante usar'a para realizar pruebas. 
\item La opci'on \texttt{marker="\,o"} indica que los puntos son representados por c'irculos. Note que, por defecto, estos puntos son unidos por rectas. Otros s'imbolos (``markers'') disponibles son listados en la tabla \ref{t}. Por ejemplo, la opci'on \texttt{marker="s"} indica al comando \texttt{plot} que grafique cuadrados. 
Adem'as, la opci'on \texttt{color} puede adoptar los valores \texttt{blue} (b), \texttt{green} (g), \texttt{red} (r), \texttt{cyan} (c), \texttt{magenta} (m), \texttt{yellow} (y), \texttt{black} (k) y \texttt{white} (w). Puede encontrar m'as colores listados \href{http://matplotlib.org/examples/color/named_colors.html}{aqu\'i}. Cambie los colores y s'imbolos del grafico en \texttt{g2.py} para familiarizarse con estas opciones.

\item Agregue una grilla (malla) a su gr'afico usando el comando \texttt{plt.grid(True)} antes de \texttt{np.savefig}, y vea qué efecto tiene esto sobre el gráfico creado.
\item Cambie los l'imites del gr'afico agregando los comandos

\begin{minted}[bgcolor=bg]{python}
plt.xlim(0,90)
plt.ylim(0,15)
\end{minted}
y vea el cambio que produce.


\item Exporte el gráfico anterior directamente a formato \texttt{.png}, simplemente cambiando \newline \texttt{plt.savefig("g2.pdf")} por \texttt{plt.savefig("g2.png")} en su programa \texttt{g2.py}. Comparta su lindo gráfico personalizado subiendo el archivo \texttt{p2.png} al foro de Teams de la práctica.

\begin{table}
\begin{center}
\begin{tabular}{cc}
\verb|"."|	& point \\
\verb|","| & pixel \\
\verb|"o"|	& circle \\
\verb|"v"|	& \verb|triangle_down| \\
\verb|"^"|	& \verb|triangle_up| \\
\verb|"<"|	& \verb|triangle_left| \\
\verb|">"|	& \verb|triangle_right| \\
\verb|"1"|	& \verb|tri_down| \\
\verb|"2"|	& \verb|tri_up| \\
\verb|"3"|	& \verb|tri_left| \\
\verb|"4"|	& \verb|tri_right| \\
\verb|"8"|	& octagon \\
\verb|"s"|	& square \\
\verb|"p"|	& pentagon \\
\verb|"*"|	& star \\
\verb|"h"|	& hexagon1 \\
\verb|"H"|	& hexagon2 \\
\verb|"+"|	& plus \\
\verb|"x"|	& x \\
\verb|"D"|	& diamond \\
\verb|"d"|	& \verb|thin_diamond| 
\end{tabular}
\caption{Algunos s'imbolos disponibles para gr'aficar puntos con el comando \texttt{plot}. Ver \href{http://matplotlib.org/api/markers_api.html}{este link} para m'as detalles y s'imbolos.}
\label{t}
\end{center}
\end{table}

\end{questions}

\end{document} 