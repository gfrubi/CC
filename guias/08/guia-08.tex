\documentclass[11pt]{exam}
\usepackage[activeacute,spanish]{babel} % Permite el idioma espa\~nol.
\usepackage[utf8]{inputenc}
\usepackage{amsmath,epsfig}
\usepackage[colorlinks]{hyperref}

\usepackage{minted} 
\usemintedstyle{emacs}
\usepackage{tcolorbox} % colores para el fondo
\definecolor{bg}{rgb}{0.95,0.95,0.95} %color de fondo

\pagestyle{headandfoot}
\spanishdecimal{.}

\begin{document}

\firstpageheadrule
\runningheadrule
\lhead{\bf\normalsize Computaci\'on Cient\'ifica\\ Gu\'ia 08}
\rhead{\bf\normalsize Cs. F\'is., Astro., Geof\'is. \\ 2018-1}
\chead{\bf\normalsize Depto. de F\'isica \\ Universidad de Concepci\'on}
\cfoot{ }
\lfoot{\tiny GR}
\begin{flushleft}
\vspace{0.2in}

\vspace{0.25cm}
\end{flushleft}
%%%%%%%%%%%%%%%%%%%%%%%%%%%%%%%%%%%%%%%%%%

\begin{questions}

\item Otro concepto muy importante en Python es el de \textit{listas}. Las listas son  similares a las cadenas, excepto que cada elemento puede ser de un tipo diferente. La sintaxis para crear listas en Python es [..., ..., ...]. Por ejemplo, ejecute:

\begin{minted}[bgcolor=bg]{python}
lista = [1, "hola", 1.0, 1-1j, True]
type(lista)
print(lista)
\end{minted}

Como puede ver, la variable \texttt{lista} es un nuevo tipo de objeto: `list'. En este caso, es una lista cuyos elementos son un entero, un string, un float, un complejo, y un booleano. Para verificar esto, imprima el valor y el tipo de cada elemento de la lista. Por ejemplo,

\begin{minted}[bgcolor=bg]{python}
print(lista[0],type(lista[0]))
print(lista[1],type(lista[1]))
\end{minted}

Este ejemplo tambi'en muestra que los 'indices de cada elemento de la lista son numerados de la misma manera que en un string:

\begin{minted}[bgcolor=bg]{python}
print(lista[0:3])
print(lista[::2])
\end{minted}

\item Los elementos de una lista pueden tener cualquier tipo reconocido por Python, por ejemplo, pueden ser otra lista!:

\begin{minted}[bgcolor=bg]{python}
superlista = ["cool",lista]
print(superlista)
\end{minted}

Imprima el valor y el tipo de cada elementos de esta lista. ?`Cu'antos elementos tiene la lista \texttt{superlista}? (respuesta, use la funci'on \texttt{len()}).

\item Existen diversas funciones en Python que crean listas. La funci'on \texttt{list()} crea una lista, por ejemplo, a partir de un string. Usando el string \texttt{x} definido anteriormente, ejecute

\begin{minted}[bgcolor=bg]{python}
y = list(x)
print(y)
print(type(y))
\end{minted}

\item Otra funci'on que crea listas 'utiles, esta vez de n'umeros \textit{enteros}, es \texttt{range(inicio,fin,paso)}, que crea una lista de valores desde \texttt{inicio} (cerrado) hasta \texttt{fin} (abierto!!), con paso \texttt{paso}. Ejecute,

\begin{minted}[bgcolor=bg]{python}
z = range(2,26,3)
print(z)
\end{minted}

Nuevamente, \texttt{(inicio,fin,paso)} funcionan de forma similar a los 'indices de un string o una lista\footnote{En Python 3, para imprimir la lista generada por \texttt{range} es necesario agregar el comando \texttt{list}, por ejemplo \texttt{print(list(range(2,26)))}}

\begin{minted}[bgcolor=bg]{python}
print(range(2,26))
print(range(26,2,-1))
\end{minted}

\textbf{Bonus track}: 
?`Qu'e hacen los siguientes comandos?, ?`Modifican el valor de \texttt{x} y/o \texttt{lista}?

\begin{minted}[bgcolor=bg]{python}
x.split(" ")
x.split("e")
lista.append("chao")
lista.insert(2,"cool")
\end{minted}

\item Otro concepto fundamental (en todo lenguaje de programaci'on) es el de \textit{ciclos}, es decir, comandos que definen tareas que se repiten un cierto n'umero de veces. En Python, una forma simple de definir un ciclo es usando el comando \texttt{for}, cuya sintaxis general es de la forma:
\begin{minted}[bgcolor=bg]{python}
for variable in lista:
	comando1
	comando2
comando_fuera_del_ciclo
\end{minted}

Un c'odigo de este tipo repite los comandos ``indentados'', es decir, escritos m'as a la derecha que los anteriores ya sea pulsando la tecla TAB o bien con cuatro espacios (en este caso, \texttt{comando1} y \texttt{comando2}) tantas veces como elementos tenga la lista \texttt{lista}. La primera ocasi'on que se ejecutan estos comandos la variable \texttt{variable} toma el valor \texttt{lista[0]}, la segunda vez el valor \texttt{lista[1]}, etc., hasta la 'ultima repetici'on donde \texttt{variable} toma el valor correspondiente al 'ultimo elemento de la lista \texttt{lista} (es decir, \texttt{lista[-1]}). 

Por ejemplo, el c'odigo siguiente:

\begin{minted}[bgcolor=bg]{python}
for palabra in ["computación", "científica", "con", "Python", 2018]:
    print(palabra)
print("terminamos con el ciclo")
\end{minted}

imprime cada uno de los elementos de la lista \texttt{[`computación', `científica', `con', `Python', 2018]}. Luego de completar el ciclo, el programa imprime el string \texttt{`terminamos con el ciclo'}. Verifique lo anterior escribiendo este c'odigo en un archivo .py y ejecut'andolo en la terminal.

\item Escriba ahora un programa almacenado en un archivo .py con el siguiente c'odigo:

\begin{minted}[bgcolor=bg]{python}
for x in range(-5,5):
    print(x)
    print(x**2)
\end{minted}

Verifique que entiende qu'e tareas realiza este sencillo programa, y por qu'e.

\item Una part'icula realiza un movimiento vertical bajo la influencia de la gravedad de modo que su altura $z(t)$ respecto al suelo es dado por la siguiente ecuaci'on de la trayectoria,
\begin{equation}
z(t)=z_0 + v_0 t -\frac{1}{2}gt^2,
\end{equation}
con $g=9.8 {\ \rm m/s^2}$. Considere el caso en que $z_0=1 {\ \rm m}$ y $v_0=24 {\ \rm m/s}$.

Escriba un programa en Python que, usando un ciclo \texttt{for}, calcule e imprima el valor de la altura $z(t)$ para los siguientes valores de tiempo (en segundos): $t=0, 0.1, 0.2, \dots 5.0$ (51 valores distintos de tiempo).

\item Modifique el programa anterior, para que ahora 'este pregunte al usuario los valores de $z_0$ y $v_0$. Para esto, use el comando \texttt{input} que aprendi'o en su trabajo con la gu'ia 07.

\item Escriba un programa en Python (que use un ciclo \texttt{for}) que calcule e imprima la suma de los primeros 1000 n'umeros enteros, es decir, el valor de 
\begin{equation}
1 + 2 + 3 + 4  + \cdots + 999 + 1000.
\end{equation}

\end{questions}


\end{document} 