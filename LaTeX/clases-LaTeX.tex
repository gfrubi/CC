\documentclass[hyperref={colorlinks},aspectratio=169]{beamer}

\usepackage{graphicx}
\usepackage{amsfonts,amsmath,color}       
\usepackage[spanish,activeacute]{babel}
\usepackage[utf8]{inputenc}

\usefonttheme{professionalfonts}
\usetheme{Darmstadt}
% Bergen, Boadilla, Copenhagen, Dresden, Hannover, Luebeck, AnnArbor, Berkeley, Darmstadt, Frankfurt, Ilmenau,     
% Madrid, Warsaw, Antibes, Berlin, CambridgeUS, Malmoe, PaloAlto
%\setBeamercovered{transparent}

\usepackage{minted} 
\usemintedstyle{emacs}

\begin{document}

\title[\LaTeX]{Introducción a \LaTeX}
\author[G.R.]{{Guillermo F. Rubilar} \\ \tiny (Basado en el Tutorial de \LaTeX ,\\
por Juan Antonio Navarro Pérez, \\Universidad de las Américas - Puebla)}
\frame{\titlepage}

%%%%%%%%%%%%%%%%%%%%%%%%%%%

\begin{frame}
\frametitle{Contenidos}
\tableofcontents
\end{frame}

%%%%%%%%%%%%%%%%%%%%%%%%%%%

\section{Introducción}
\begin{frame}[fragile]\frametitle{?`\TeX{} y \LaTeX?}
\begin{itemize}
\item \href{https://es.wikipedia.org/wiki/TeX}{\TeX{}} (1978) es un sistema profesional de \emph{composición tipográfica} desarrollado
por \href{https://es.wikipedia.org/wiki/Donald_Knuth}{Donald E. Knuth} (1938, prof. emérito U. Stanford).
\item \TeX{} fue dise\~nado para producir documentos (especialmente con expresiones matemáticas) con la más alta \emph{calidad de imprenta}.
\item \href{https://es.wikipedia.org/wiki/LaTeX}{\LaTeX}\ es un \emph{sistema de macros}, desarrollado sobre \TeX{} por \href{https://es.wikipedia.org/wiki/Leslie_Lamport}{Leslie Lamport} (1983), para facilitar su uso por parte de los autores.
\end{itemize}
\end{frame}

\begin{frame}[fragile]\frametitle{?`\TeX{} y \LaTeX?}
\begin{itemize}
\item \href{https://es.wikipedia.org/wiki/AMS-LaTeX}{\AmS -\LaTeX} es un conjunto de paquetes \LaTeX\ para matemáticas desarrollado por la American Mathematical Society. Disponible en \LaTeX\ como \texttt{amsmath} 
(1990).
  \item Versión actual: \LaTeX2e (1994). Código fuente en \href{https://github.com/latex3/latex2e}{GitHub}.
  \item El futuro: proyecto \href{https://www.latex-project.org/latex3/}{\LaTeX3}. Código fuente en \href{https://github.com/latex3/latex3}{GitHub}.
	\begin{center}
		\includegraphics[height=3cm]{figs/tux26.pdf}\includegraphics[height=2.5cm]{figs/latex-project-logo.pdf}
	\end{center}
\end{itemize}
\end{frame}


\begin{frame}[fragile]\frametitle{Word/Writer vs \LaTeX}
\begin{center}
\begin{tabular}{p{0.4\textwidth}p{0.4\textwidth}}
\multicolumn{1}{c}{Word/Writer	}     & \multicolumn{1}{c}{\LaTeX}      \\
\\

$\bullet$ WYSIWYG            & $\bullet$ Preprocesado          \\
$\bullet$ Muy fácil de usar & $\bullet$ No siempre fácil     \\
$\bullet$ Facilidades para insertar objetos 
                             & $\bullet$ Limitaciones por formatos de archivo \\[1ex]
$\bullet$ Lento y malo para tratar expresiones matemáticas
                             & $\bullet$ Muy bueno para expresiones matemáticas \\
$\bullet$ 'Enfasis en Dise\~no & $\bullet$ Énfasis en Contenido         
\end{tabular}
\end{center}
\end{frame}

\begin{frame}[fragile]\frametitle{?`Por qu'e usar \LaTeX?}
\begin{itemize}
\item Produce documentos con calidad de imprenta.
\item Utilizado por editoriales, revistas y congresos especializados.
\item Indispensable para f'isic@s, geof'isic@s, astrónom@s, matemátic@s, etc.
\item Es la mejor opción para escribir su \emph{tesis}!.
\end{itemize}
\begin{center}
	\includegraphics[height=5cm]{figs/LaTeX_vs_Word.pdf}
\end{center}
\end{frame}


\begin{frame}[fragile]\frametitle{Filosof'ia de \LaTeX}

\bigskip

\bigskip
\begin{block}{}
La persona que escribe debe de preocuparse del \textit{contenido} de sus documentos, y no (directamente) de la \textit{apariencia} que 'estos tendrán en el resultado final.
\end{block}
\end{frame}

\section{Edición Básica}


\begin{frame}[fragile]\frametitle{Mi primer documento}
\begin{block}{}
\begin{minted}{tex}
\documentclass{article}
\author{Nombre de Autor(a)}
\title{Mi Primer Documento}

\begin{document}
\maketitle

Hola. Este es mi primer documento.
\end{document}
\end{minted}
\end{block}
\end{frame}

%\begin{frame}[fragile]\frametitle{Proceso de compilación}
%\vspace{0.6cm}
%\begin{center}
%	\scalebox{0.8}{\input{compilado.tex}}
%\end{center}
%\end{frame}

%\begin{frame}[fragile]\frametitle{Obtención de PDF}
%\vspace{0.6cm}
%\begin{center}
%	\scalebox{0.8}{\input{compilado2.tex}}
%\end{center}
%\end{frame}

\begin{frame}[fragile]\frametitle{Proceso de compilación}

\begin{block}{Forma tradicional}
\begin{itemize}
\item Compilar: \par
\texttt{> latex archivo.tex}
%\item Pre-visualizar: \par
%\texttt{> xdvi archivo.dvi}
%\item Generar Post-Script: \par
%\texttt{> dvips archivo.dvi -o archivo.ps}
%\item Imprimir: \par
%\texttt{> lpr -Plaser1sala4 archivo.ps}
\item Convertir archivo .dvi a Pdf: \par
\texttt{> dvipdf archivo.dvi}\par
%\texttt{> ps2pdf archivo.ps archivo.pdf}
\end{itemize}
\end{block}
\begin{block}{Forma rápida (Recomendada)}
\begin{itemize}
\item Compilar directamente a pdf: \par
\texttt{> pdflatex archivo.tex}
\end{itemize}
\end{block}
\end{frame}


\begin{frame}[fragile]\frametitle{Clases de documentos}
Clases estándares
\begin{itemize}
  \item \texttt{article} -- Art'iculo.
  \item \texttt{report} -- Reporte.
  \item \texttt{book} -- Libro.
  \item \texttt{letter} -- Cartas.
\end{itemize}
Clases extras
\begin{itemize}
  \item \texttt{beamer} -- Presentaciones.
  \item \texttt{prosper} -- Presentaciones.
  \item \texttt{poster} -- Poster.
\end{itemize}
\end{frame}

\begin{frame}[fragile]\frametitle{Unidades estructurales}
Para libros y reportes:
\begin{itemize}
  \item \verb|\part{...}|
  \item \verb|\chapter{...}|
\end{itemize}
Para libros, art'iculos y reportes:
\begin{itemize}
  \item \verb|\section{...}|
  \item \verb|\subsection{...}|
  \item \verb|\subsubsection{...}|
\end{itemize}
'Indice: \verb|\tableofcontents|.
\end{frame}

\begin{frame}[fragile]\frametitle{Listas con Vi\~netas}
\begin{block}{}
\begin{minted}{tex}
  \begin{itemize}
    \item Un elemento de la lista.
    \item Otro elemento de la lista.
  \end{itemize}
 \end{minted}
\end{block}
  
    \begin{itemize}
    \item Un elemento de la lista.
    \item Otro elemento de la lista.
  \end{itemize}
\end{frame}

\begin{frame}[fragile]\frametitle{Listas Enumeradas}
\begin{block}{}
\begin{minted}{tex}
  \begin{enumerate}
    \item El primer elemento de la lista.
    \item El segundo elemento de la lista.
  \end{enumerate}
 \end{minted}
\end{block}
  
   \begin{enumerate}
    \item El primer elemento de la lista.
    \item El segundo elemento de la lista.
  \end{enumerate}
\end{frame}

\begin{frame}[fragile]\frametitle{Listas Anidadas}
  \begin{enumerate}
    \item El primer elemento de la lista.
    \begin{enumerate}
      \item Un sub elemento.
      \item El segundo sub elemento.
    \end{enumerate}
    \item El segundo elemento de la lista.
    \begin{itemize}
      \item Con algunos puntos \dots
      \item \dots importantes.
    \end{itemize}
    \item Y el \'ultimo elemento.
  \end{enumerate}
\end{frame}

\begin{frame}[fragile]\frametitle{Listas Anidadas}
\begin{block}{}
\begin{minted}{tex}
  \begin{enumerate}
    \item El primer elemento de la lista.
    \begin{enumerate}
      \item Un sub elemento.
      \item El segundo sub elemento.
    \end{enumerate}
    \item El segundo elemento de la lista.
    \begin{itemize}
      \item Con algunos puntos \dots
      \item \dots importantes.
    \end{itemize}
    \item Y el \'ultimo elemento.
  \end{enumerate}
\end{minted}
\end{block}
\end{frame}

\begin{frame}[fragile]\frametitle{Citas Textuales}
\dots como dijo alguien muy sabio,
\begin{quote}
``The dark side of the Force is a pathway to many abilities, some considered to be unnatural''
\end{quote}
mientras miraba a su futuro aprendiz.

\begin{block}{}
\begin{minted}{tex}
\dots como dijo alguien muy sabio,
\begin{quote}
``The dark side of the Force is a pathway to many 
abilities, some considered to be unnatural''
\end{quote}
mientras miraba a su futuro aprendiz.
\end{minted}
\end{block}
\end{frame}


\begin{frame}[fragile]\frametitle{Texto Enfatizado}

Decimos que un n\'umero es \emph{racional} si existen dos enteros \dots

\begin{block}{}
\begin{minted}{tex}
Decimos que un n\'umero es \emph{racional} si existen 
dos enteros \dots
\end{minted}
\end{block}

\begin{itemize}
\item \verb|\emph{...}| enfatiza parte del texto.
\item \emph{!`Piensa en contenido, no en formato!}
%\item Los t'erminos nuevos en definiciones usualmente se enfatizan.
\end{itemize}
\end{frame}


\begin{frame}[fragile]\frametitle{Notas al pie de página}
\begin{block}{}
\begin{minted}{tex}
Uno de los grandes personajes de la F\'isica sin duda 
es Sir Isaac Newton\footnote{Isaac Newton: 25 de 
diciembre de 1642 (jul.) / 4 de enero de 1643 (greg) 
-- 20 de marzo (jul.) / 31 de marzo de 1727 (greg.) 
fue un f\'isico, filósofo, teólogo, inventor, 
alquimista y matemático ingl\'es.} quien, entre 
otras cosas, desarrolló los fundamentos de la 
\emph{Mecánica}.
\end{minted}
\end{block}

Uno de los grandes personajes de la F\'isica sin duda es 
Sir Isaac Newton\footnote{Isaac Newton: 25 de diciembre de 
1642 (jul.) / 4 de enero de 1643 (greg) -- 20 de marzo 
(jul.) / 31 de marzo de 1727 (greg.) fue un f\'isico, 
filósofo, teólogo, inventor, alquimista y matemático 
ingl\'es.} quien, entre otras cosas, desarrolló los 
fundamentos de la \emph{Mecánica}.
\end{frame}


\begin{frame}[fragile]\frametitle{Comandos de Formato}

\begin{center}
\begin{tabular}{ll}
\verb|\textrm{}|  & \textrm{Romano} \\
\verb|\textsf{}|  & \textsf{Serif} \\
\verb|\texttt{}|  & \texttt{Typewriter} \\
\verb|\textbf{}|  & \textbf{Negritas} \\
\verb|\textit{}|   & \textit{Itálicas} \\
\verb|\textsl{}|   & \textsl{Slanted} \\
\verb|\textsc{}|  & \textsc{Small Caps} \\
\verb|\underline{}|  & \underline{Subrayado} \\
\end{tabular}
\end{center}

Hay versiones \verb|\mathXX{}| equivalentes para modo matemático.
Y \verb|\mathcal{}| $\mathcal{CAL}$.
\end{frame}

\begin{frame}[fragile]\frametitle{Tama\~no de Letra}
\begin{center}
\begin{tabular}{ll}
\verb|{\tiny }| & {\tiny Peque\~nita} \\
\verb|{\scriptsize}| & {\scriptsize scriptsize}\\
\verb|{\footnotesize}| & {\footnotesize tama\~no de nota al pie}\\
\verb|{\small }| & {\small Peque\~na} \\
\verb|{\normalsize }| & {\normalsize Normal} \\
\verb|{\large }| & {\large Grande} \\
\verb|{\Large }| & {\Large Grandota} \\
\verb|{\LARGE }| & {\LARGE Grandototota} \\
\verb|{\huge }| & {\huge Enorme} \\
\verb|{\Huge }| & {\Huge Mega Enorme} \\
\end{tabular}
\end{center}
\end{frame}

\begin{frame}[fragile]\frametitle{Comandos de Alineación}
\begin{itemize}
\item \verb|\begin{center}| \par \verb|\end{center}|
\item \verb|\begin{flushleft}| \par \verb|\end{flushleft}|
\item \verb|\begin{flushright}| \par \verb|\end{flushright}|
\item \verb|\begin{sloppypar}| \par \verb|\end{sloppypar}|
\end{itemize}
\end{frame}

\begin{frame}[fragile]\frametitle{Espa\~nol y \LaTeX}
Forma tradicional
  \begin{center}
  \begin{tabular}{c|c}
     Input      &   Resultado \\ 
     \hline
     \verb+ó+  &   ó       \\
     \verb+\'u+  &   'u       \\
     \verb+á+  &   á       \\
     \verb+\'i+  &   'i        \\
     \verb+\~n+  &   \~n     \\
     \verb+\~N+  &   \~N     \\
     \verb+?`+  &   ?`     \\
     \verb+!`+  &   !`  
  \end{tabular}
\end{center}
\end{frame}



\begin{frame}[fragile]\frametitle{Reglas generales de edición}
\begin{itemize}
\item Usar espacios para separar \emph{palabras}.
\item Un espacio vale igual que mil.
\item Los fines de línea sencillos no valen.
\end{itemize}
\begin{itemize}
\item Usar líneas vac'ias para separar \emph{párrafos}.
\item Una línea vac'ia vale igual que mil.
\end{itemize}
\begin{itemize}
\item El espaciado y las sangr'ias son trabajo de \LaTeX, y lo
  sabe hacer muy bien.
\item \emph{No forzar espacios ni cortes de l'inea.}
\end{itemize}
\end{frame}



\section{Matemáticas con \LaTeX}

\begin{frame}[fragile]\frametitle{Fórmulas en l'inea}
Las fórmulas en l'inea ocurren dentro de la secuencia natural de un párrafo.

\begin{block}{}
\begin{minted}{tex}
Sea $x$ un número real en el intervalo $(0, 1)$. 
Observe también que $0 < x^2 < 1$.
\end{minted}
\end{block}

\begin{quote}
Sea $x$ un número real en el intervalo $(0, 1)$. 
Observe también que $0 < x^{2} < 1$.
\end{quote}

\end{frame}

\begin{frame}[fragile]\frametitle{Fórmulas en l'inea}
\begin{itemize}
  \item Los signos \verb|$ $| indican el contenido matemático.
  \item Todo el contenido matemático (y sólo el contenido matemático) debe ser marcado.
  \item No usar el contenido matemático para poner itálicas.
  \item Y no usar comandos de formato para marcar contenido matemático.
  \item Pensar en el contenido, \emph{!`no en el formato!}.
\end{itemize}
\end{frame}


\begin{frame}[fragile]\frametitle{S'imbolos Especiales}
\begin{itemize}
  \item Letras griegas min'usculas
   \begin{center}
      \begin{tabular}{clcl}
        $\alpha$ & \verb+\alpha+ & $\theta$   & \verb+\theta+    \\
        $\beta$  & \verb+\beta+  & $\vartheta$& \verb+\vartheta+ \\
        \multicolumn{4}{c}{\ldots}                               \\
        $\lambda$& \verb+\lambda+& $\varsigma$& \verb+\varsigma+ 
      \end{tabular}
    \end{center}
   \item Legras griegas may'usculas
   \begin{center}
   \begin{tabular}{llcllcll}
   $\Gamma $ & \verb|\Gamma| & & $\Xi $ & \verb|\Xi| & &$\Phi $ & \verb|\Phi|  \\
$\Delta $ & \verb|\Delta| & & $\Pi $ & \verb|\Pi| & & $\Psi $ & \verb|\Psi|  \\
$\Theta $ & \verb|\Theta| & &$\Sigma $ & \verb|\Sigma| & & $\Omega $ & \verb|\Omega|  \\
$\Lambda $ & \verb|\Lambda| & & $\Upsilon $ & \verb|\Upsilon|  \\
  \end{tabular}
  \end{center}
  \end{itemize}
  Ver 
    \begin{center}
	\url{https://es.wikipedia.org/wiki/Alfabeto_griego}.
  \end{center}
\end{frame}

\begin{frame}[fragile]\frametitle{S'imbolos Especiales}
\begin{itemize}
\item Operaciones binarias
      \begin{center}
       \begin{tabular}{clcl}
          $\pm$      & \verb+\pm+      &  $\mp$    & \verb+\mp+        \\
          $\cdot$    & \verb+\cdot+    &  $\bullet$& \verb+\bullet+    \\
          $\setminus$& \verb+\setminus+&  $\cap$   & \verb+\cap+
       \end{tabular}
    \end{center}
  \item Acentos matemáticos
  \begin{center}
  \begin{tabular}{llcll}
   \verb|\hat a| & $\hat a$ & &\verb|\check a| & $\check a$ \\
 \verb|\tilde a| & $\tilde a$ & &\verb|\acute a| &  $\acute a$ \\
 \verb|\grave a| & $\grave a$ & & \verb|\dot a|  & $\dot a$ \\
 \verb|\ddot a|  & $\ddot a$  & & \verb|\breve a| & $\breve a$ \\
 \verb|\bar a|   & $\bar a$  & & \verb|\vec a|   & $\vec a$
 \end{tabular}
 \end{center}
 \end{itemize}
\end{frame}

\begin{frame}[fragile]\frametitle{S'imbolos Especiales}
\begin{itemize}
  \item S'imbolos diversos
   \begin{center}
  \begin{tabular}{llcll}
  $\aleph $ & \verb|\aleph| & & $\prime $& \verb|\prime| \\
$\forall $& \verb|\forall|  & & $\hbar $& \verb|\hbar| \\
$\emptyset $ & \verb|\emptyset| & & $\exists $& \verb|\exists|  \\
$\imath $ & \verb|\imath| & & $\nabla $& \verb|\nabla| \\
$\neg $& \verb|\neg|  & & $\jmath $& \verb|\jmath| \\
$\surd $& \verb|\surd| & & $\flat $& \verb|\flat| \\
$\ell $& \verb|\ell| & &$\top $& \verb|\top| \\
\end{tabular}
 \end{center}
  \end{itemize}
\end{frame}

\begin{frame}[fragile]\frametitle{S'imbolos Especiales}
\begin{center}
  	\begin{tabular}{llcll}
	$\natural $ & \verb|\natural| & & $\wp $& \verb|\wp| \\
	$\bot $ & \verb|\bot| & & $\sharp $& \verb|\sharp| \\
	$\Re $& \verb|\Re| & & $\Vert $& \verb.\|. \\
	$\clubsuit $& \verb|\clubsuit| & & $\Im $&\verb|\Im| \\
	$\diamondsuit $&  \verb|\diamondsuit| & & $\partial $& \verb|\partial| \\
	$\triangle $& \verb|\triangle| & & $\heartsuit $& \verb|\heartsuit| \\
	$\infty $& \verb|\infty| & & $\backslash $& \verb|\backslash| \\
	$\spadesuit $& \verb|\spadesuit| & & $\mho $& \verb|\mho| \\
	$\Box $& \verb|\Box| & &$\Diamond $& \verb|\Diamond| \\
	$\angle $ & \verb|\angle| && &\\
  	\end{tabular}

	Ver 

	\url{https://latex.wikia.org/wiki/List_of_LaTeX_symbols}.

\end{center}
\end{frame}

\begin{frame}[fragile]\frametitle{S'imbolos Especiales}
  \begin{itemize}
   \item Nombres de funciones de uso com'un: \verb|\sin|, \verb|\cos|, \verb|\log|, \verb|\lim|, \dots
   \item Algunos comandos t'ipicos:
   
\begin{center}
  \begin{tabular}{ll}
    \verb|\sqrt{2}| & $\sqrt{2}$ \\
    \verb|x \leq 4| & $x \leq 4$ \\
    \verb|\frac{1}{3+i}| & $\frac{1}{3+i}$ \\
  \end{tabular}
\end{center}
	
	\item Caracteres especiales (reservados en \LaTeX): \verb|$ & % # _ ^ { } ~ \| se generan usando \verb+\$ \& \% \# \_ \verb|^| \{ \} \verb|~| y \verb|\|+
	
  \end{itemize}
\end{frame}

\begin{frame}[fragile]\frametitle{Exponentes y sub'indices}
\begin{itemize}
  \item Exponentes: \verb|x^2|: $x^2$
  \item Sub'indices: \verb|x_i|: $x_i$
  \item Para usar exponentes y sub'indices de más de un caracter, usar \verb|{}|. Ejemplos
  \begin{center}
  \begin{tabular}{ll}
    \verb|x^{2\pi}|      & $x^{2\pi}$ \\
    \verb|x_{i+1}|       & $x_{i+1}$ \\
    \verb|x_{i+1}^{2}|   & $x_{i+1}^{2}$ \\
    \verb|x_{(i+1)^{2}}| & $x_{(i+1)^{2}}$ \\
  \end{tabular}
  \end{center}
\end{itemize}
\end{frame}

\begin{frame}[fragile]\frametitle{L'imites y sumatorias}
\begin{itemize}
  \item Comandos: \verb|\lim|, \verb|\sum|, \verb|\int|
  \item Ejemplos:
\end{itemize}
  \begin{center}
  \begin{tabular}{ll}
    \verb|\lim_{x \to 0} \sin(x)/x|      & \quad $\lim_{x \to 0} \sin(x)/x$ \\[2pt]
    \verb|\sum_{i=0}^n i^{2}|          & \quad $\sum_{i=0}^{n} i^{2}$     \\[2pt]
    \verb|F(x) = \int_0^1 f(x)\,dx|  & \quad $F(x) = \int_{0}^{1} f(x)\,dx$ \\
  \end{tabular}
  \end{center}
\end{frame}

\begin{frame}[fragile]\frametitle{Entorno ``equation''}
\begin{block}{}
\begin{minted}{tex}
La suma de cuadrados
\begin{equation}
  \sum_{i=0}^n i^2
\end{equation}
tiene una fórmula muy sencilla.
\end{minted}
\end{block}

\begin{quote}
La suma de cuadrados
\begin{equation}
  \sum_{i=0}^{n} i^{2}
\end{equation}
tiene una fórmula muy sencilla.
\end{quote}
\end{frame}

\begin{frame}[fragile]\frametitle{Entorno ``equation''}
\begin{block}{}
\begin{minted}{tex}
\dots y después de muchos cálculos llegamos a la
inevitable conclusión que 
\begin{equation}
  \lim_{x \to 0} \frac{\sin(x)}{x} = 1.
\end{equation}

Pasando a otros temas \dots
\end{minted}
\end{block}


\begin{quote}
\dots y despu\'es de muchos cálculos llegamos a la
inevitable conclusión que 
\begin{equation}
  \lim_{x \to 0} \frac{\sin(x)}{x} = 1.
\end{equation}

Pasando a otros temas \dots
\end{quote}
\end{frame}

\begin{frame}[fragile]\frametitle{Notas de Redacción}
\begin{itemize}
  \item Las fórmulas deben ocurrir de manera natural dentro de la lectura 
    de un párrafo (las ecuaciones se leen como parte del texto!).
%  \item Recuerda los signos de puntuación. Utiliza los comandos 
%     \verb|\,,| o \verb|\,.| al final de una fórmula en modo display si es necesario.
  \item No dejar líneas en blanco entre los comandos \verb|\begin{equation}|,
   \verb|\end{equation}| y el resto de las l'ineas del párrafo. Recuerda que la
   fórmula \emph{forma parte} del párrafo.
\item \LaTeX\ numera automáticamente las ecuaciones!.
\item En ocasiones es conveniente agregar peque\~nos espacios:
	\begin{itemize}
		\item \verb|\,| espacio delgado: $\int f(x)\,dx$ (\verb|$\int f(x)\,dx$|).
		\item \verb|\;| espacio ancho: $\int f(x)\; dx$ (\verb|$\int f(x)\; dx$|).
		\item \verb|\ | espacio normal: $\int f(x)\ dx$ (\verb|$\int f(x)\  dx$|).
		\item \verb|\quad| espacio grande: $\int f(x)\quad dx$ (\verb|$\int f(x)\quad  dx$|).
		\item \verb|\qquad| espacio más grande: $\int f(x)\qquad dx$ (\verb|$\int f(x)\qquad  dx$|)
	\end{itemize}
\end{itemize}
\end{frame}



\begin{frame}[fragile]\frametitle{Arreglos y matrices}

\begin{block}{}
\begin{minted}{tex}
\begin{equation}
  \left(\begin{array}{ccc}
     \cos\theta & \sin\theta & 0 \\
    -\sin\theta & \cos\theta & 0 \\
     T_x       & T_y       & 1
  \end{array}\right)
\end{equation}
\end{minted}
\end{block}


\begin{equation}
  \left(\begin{array}{ccc}
     \cos\theta & \sin\theta & 0 \\
    -\sin\theta & \cos\theta & 0 \\
     T_x       & T_y       & 1
  \end{array}\right)
\end{equation}

\end{frame}

\begin{frame}[fragile]\frametitle{Arreglos y matrices}
\begin{itemize}
  \item Los comandos \verb|\left| y \verb|\right| ponen par'entesis que se adaptan al tama\~no del conenido que encierran. Se pueden usar combinaciones de: \verb|(|, \verb|)|, \verb|[|, \verb|]|,    \verb|\{|, \verb|\}|, \verb#|#, \verb|.|, \dots
  \item El entorno \verb|array| recibe una lista de las columnas del arreglo,
     una letra: \verb|l| (left), \verb|c| (center), \verb|r| (right) para
     indicar la alíneación de cada columna.
  \item Las columnas se separan con \verb|&| y los renglones con \verb|\\|.
\end{itemize}
\end{frame}


\begin{frame}[fragile]\frametitle{Funciones por partes}
\begin{equation}
  f(x) = \left\{
    \begin{array}{ll}
      x,     & -\infty \leq x \leq 1 \\
      1 - x,  & 1 \leq x \leq 2 \\
      0,      & x > 2
    \end{array}\right.
\end{equation}
\end{frame}

\begin{frame}[fragile]\frametitle{Funciones por partes}
\begin{block}{}
\begin{minted}{tex}
\begin{equation}
  f(x) = \left\{
    \begin{array}{ll}
      x,     & -\infty \leq x \leq 1 \\
      1 - x,  & 1 \leq x \leq 2 \\
      0,      & x > 2
    \end{array}\right.
\end{equation}
\end{minted}
\end{block}

\begin{itemize}
  \item \verb|\right.| coloca un delimitador invisible (para cerrar el par\'entesis corchete).
\end{itemize}
\end{frame}

\begin{frame}[fragile]\frametitle{Extendiendo \LaTeX\ paquetes/módulos adicionales}
\begin{block}{}
Las funcionalidades de \LaTeX\ pueden ampliarse casi indefinidamente cargando \textit{paquetes}. Existen cientos (quizás miles) de paquetes disponibles. Una lista de los principales paquetes, ordenada alfab\'eticamente, puede encontrarse en \href{https://ctan.org/pkg/}{este link}.
\end{block}

\begin{block}{Cargando paquetes}
En general, cada paquete particular que quiera ser usado en un documento debe ser cargado en el preámbulo del documento, es decir, \textit{antes} del comando \verb|\begin{document}|
\end{block}
\begin{block}{}
\begin{minted}{tex}
\usepackage[opciones]{nombre_paquete}
\end{minted}
\end{block}
\end{frame}

\begin{frame}[fragile]\frametitle{Espa\~nol y \LaTeX: Babel}
\begin{block}{}
\begin{minted}{tex}
\usepackage[spanish, activeacute]{babel}
\end{minted}
\end{block}
\begin{block}{}
El paquete Babel se encarga de gestionar los cortes de palabras al final de las l\'ineas (muy \'util!). La opción \texttt{spanish} selecciona nuestro idioma, y \texttt{activeacute} permite acortar un poco la introducción de tildes y caracteres latinos
\end{block}
\begin{block}{}
\begin{minted}{tex}
á, 'e, 'i, ó, 'u, ~n, 'N, ?` y !`
\end{minted}
producen
\begin{center}
á, 'e, 'i, ó, 'u, ~n, 'N, ?` y !`
\end{center}
\end{block}
\end{frame}



\begin{frame}[fragile]\frametitle{Espa\~nol y \LaTeX}
\begin{block}{inputenc}
\begin{minted}{tex}
\usepackage[utf8]{inputenc}
\end{minted}
permite ingresar los tildes directamente en el texto. Para usar esta opción debe tenerse el cuidado de verificar que el archivo de código \LaTeX\ est\'e almacenado en formato UTF8!.
\end{block}
\begin{block}{}
En este caso 
\begin{minted}{tex}
á, é, í, ó, ú, ñ, Ñ, ¿ y ¡
\end{minted}
producen directamente
\begin{center}
á, é, í, ó, ú, ñ, Ñ, ¿ y ¡
\end{center}
\end{block}
\end{frame}

\begin{frame}[fragile]\frametitle{\AmS-\LaTeX}
\begin{block}{}
En paquete \href{http://www.ams.org/publications/authors/tex/amslatex}{\AmS-Math}, desarrollado por la \textbf{A}merican \textbf{M}athematical \textbf{S}ociety, implementa extensiones a \LaTeX\ que facilitan la escritura de expresiones matemáticas y mejoran la apariencia del resultado final. Se carga agregando 
\begin{minted}{tex}
\usepackage{amsmath}
\end{minted}
al preámbulo del documento.
\end{block}
\end{frame}



%\begin{frame}[fragile]\frametitle{Ecuaciones muy largas}
%\begin{equation}
%  \begin{array}{rcl}
%    \Sigma & = & x_{1} + x_{2} + x_{3} + x_{4} + x_{5} + \\
%           &   & {} + x_{6} + x_{7} + x_{8} + x_{9} +    \\
%           &   & {} + x_{10} + x_{11} + x_{12} + x_{13}  \\
%           & = & \sum_{i=1}^{13} x_{i}
%  \end{array}
%\end{equation}
%
%\begin{verbatim}
%\begin{equation}
%	\begin{array}{rcl}
% 	 \Sigma & = & x_{1} + x_{2} + x_{3} + x_{4} + x_{5} + \\
%         &   & {} + x_{6} + x_{7} + x_{8} + x_{9} +    \\
%         &   & {} + x_{10} + x_{11} + x_{12} + x_{13}  \\
%         & = & \sum_{i=1}^{13} x_{i}
%	\end{array}
%\end{equation}
%\end{verbatim}
%\end{frame}

\begin{frame}[fragile]\frametitle{M\'ultiples ecuaciones alineadas}
\begin{eqnarray}
I &=&I_{\rm cm}+MD^{2} \\
&=&\frac{1}{12}ML^2 +M\left( \frac{L}{2}-\frac{L}{5}\right)^2 \\
&=&\frac{13}{75}L^2 M \\
&\approx&9,7067\times 10^{-2} [kg\,m^2] .
\end{eqnarray}
\end{frame}

\begin{frame}[fragile]\frametitle{M\'ultiples ecuaciones alineadas}
\begin{block}{}
\begin{minted}{tex}
\begin{eqnarray}
   I &=&I_{\rm cm}+MD^{2} \\
     &=&\frac{1}{12} ML^2 + M\left( \frac{L}{2}
     	-\frac{L}{5} \right)^2 \\
     &=&\frac{13}{75} L^{2} M \\
     &\approx & 9,7067 \times 10^{-2} [kg\,m^2].
\end{eqnarray}
\end{minted}
\end{block}
\end{frame}

\begin{frame}[fragile]\frametitle{M\'ultiples ecuaciones alineadas:  \texttt{align} de \texttt{amsmath}}
El paquete \texttt{amsmath} suministra el entorno \texttt{align}, con una sintaxis casi igual a \texttt{eqnarray}, pero con algunas mejoras en detalles de alineación:
\begin{align}
I &= I_{\rm cm} + MD^2 \\
&= \frac{1}{12} ML^2 + M\left(\frac{L}{2}-\frac{L}{5}\right)^2 \\
&= \frac{13}{75} L^2M \\
&\approx 9,7067 \times 10^{-2} [kg\,m^2].
\end{align}
\end{frame}

\begin{frame}[fragile]\frametitle{M\'ultiples ecuaciones alineadas:  \texttt{align} de \texttt{amsmath}}
\begin{block}{}
\begin{minted}{tex}
\begin{align}
I &= I_{\rm cm} + MD^2 \\
&= \frac{1}{12} ML^2 + M\left(\frac{L}{2}
	-\frac{L}{5}\right)^2 \\
&= \frac{13}{75} L^2M \\
&\approx 9,7067 \times 10^{-2} [kg\,m^2].
\end{align}
\end{minted}
\end{block}
\end{frame}

\section{Referencias Cruzadas}

\begin{frame}[fragile]\frametitle{Referencias Cruzadas}
 El torque resultante es la suma del torque aplicado
sobre 1 más el torque aplicado sobre 2. Es decir:
\begin{equation}
\tau_{\rm total}=\tau_1+\tau_2,  \label{Ttotal}
\end{equation}
donde
\begin{equation}
\tau_1 =r_1 F_1 \sin\theta_1,  \label{T11}
\end{equation}
es positivo ya que la rotación va en sentido anti-horario, mientras que
\begin{equation}
\tau_2 = -r_2 F_2 \sin\theta_2,  \label{T22}
\end{equation}
es negativo ya que la rotación va en sentido horario. 
Luego, reemplazando (\ref{T11}) y (\ref{T22}) en (\ref{Ttotal}), 
tendremos que \dots
\end{frame}

\begin{frame}[fragile]%\frametitle{Referencias Cruzadas}
\begin{block}{}
\begin{minted}{tex}
 El torque resultante es la suma del torque aplicado
sobre 1 más el torque aplicado sobre 2. Es decir:
\begin{equation}
\tau_{\rm total}=\tau_1+\tau_2,  \label{Ttotal}
\end{equation}
donde
\begin{equation}
\tau_1 =r_1 F_1 \sin\theta_1,  \label{T11}
\end{equation}
es positivo ya que la rotación va en sentido 
anti-horario, mientras que
\begin{equation}
\tau_2 = -r_2 F_2 \sin\theta_2,  \label{T22}
\end{equation}
es negativo ya que la rotación va en sentido 
horario. Luego, reemplazando (\ref{T11}) y (\ref{T22}) 
en (\ref{Ttotal}), tendremos que \dots
\end{minted}
\end{block}
\end{frame}


\begin{frame}[fragile]\frametitle{Referencias Cruzadas}
\begin{itemize}
\item Se puede poner \verb|\label{..}| despu'es de:
  \begin{itemize}
    \item \verb|\begin{equation}|, \verb|\begin{eqnarray}|, \dots
    \item \verb|\begin{table}|, \verb|\begin{figure}|, \dots
    \item \verb|\chapter{..}|, \verb|\section{..}|, \dots
    \item Casi cualquier cosa que numere.
  \end{itemize}
\item Se puede poner \verb|\ref{..}|:
  \begin{itemize}
    \item !`Donde quieras en el documento!
  \end{itemize}
\item Recuerda recompilar para actualizar referencias.
\item \texttt{amsmath} tambi'en suministra \verb|\eqref{..}| para citar ecuaciones, que permite reemplazar \verb|(\ref{..})| por \verb|\eqref{..}|.
\end{itemize}
\end{frame}

\begin{frame}[fragile]\frametitle{Consejos de Redacción}
  \begin{itemize}
%  \item Notación: \verb|Cap\'itulo~\ref{intro}|
%  \begin{itemize}
%    \item La palabra clave en may'uscula.
%    \item No olvides usar `\verb|~|'{} en lugar de espacio.
%  \end{itemize}
  \item Usa nombres descriptivos para las etiquetas:
  \begin{itemize}
    \item \texttt{newton}, \texttt{maxwellhom}, \texttt{solucion2}
  \end{itemize}
  \item Evita usar nombres que no te dicen nada:
  \begin{itemize}
    \item \texttt{tdmapmu}, \texttt{ec2}, \texttt{p}
  \end{itemize}
\end{itemize}
\end{frame}



\begin{frame}[fragile]\frametitle{Citas Bibliográficas}
\begin{block}{}
\begin{minted}{tex}
\begin{document}
...

Si Ud. quiere ser sec@ en Relatividad General, 
l\'ease este librito \cite{MTW73}.

...

\begin{thebibliography}{99}
\bibitem{MTW73} C.W. Misner, K.S. Thorne and J.A. 
Wheleer, {\em Gravitation},  W.H. Freeman and Company, 
San Francisco (1973).
\end{thebibliography}
\end{document}
\end{minted}
\end{block}
\end{frame}

\section{Tablas y Figuras}

\begin{frame}[fragile]\frametitle{Tablas Simples}

\begin{center}
\begin{tabular}{c|cc}
A\~no  & Ventas   & Inversión \\ \hline
1999  & \$ 3.900 &    1.4\%   \\
2000  & \$ 2.700 &    3.6\%   \\
2001  & \$ 3.200 &    2.3\%   \\
2002  & \$ 3.700 &    4.9\%   \\
2003  & \$ 4.100 &    3.4\%   \\
\end{tabular}
\end{center}

\end{frame}

\begin{frame}[fragile]\frametitle{Tablas Simples}
\begin{block}{}
\begin{minted}{tex}
\begin{center}
  \begin{tabular}{c|cc}
    A\~no  & Ventas   & Inversión \\ \hline
    1999  & \$ 3.900 &    1.4\%   \\
    2000  & \$ 2.700 &    3.6\%   \\
    2001  & \$ 3.200 &    2.3\%   \\
    2002  & \$ 3.700 &    4.9\%   \\
    2003  & \$ 4.100 &    3.4\%   \\
  \end{tabular}
\end{center}
\end{minted}
\end{block}
\end{frame}

\begin{frame}[fragile]\frametitle{Tablas Simples}
\begin{itemize}
  \item El ambiente \verb|tabular| se parece mucho a \verb|array|, pero funciona
    en modo texto.
  \item Usa barras \verb$|$ en la descripción de la columna para indicar lineas
    verticales, y el comando \verb|\hline| para l'ineas horizontales.
  \item \textbf{Sugerencia}: No agreges demasiadas l'ineas a una tabla, usa sólo las 
    necesarias para separar o distinguir los valores importantes.
\end{itemize}
\end{frame}

\begin{frame}[fragile]\frametitle{Multicolumnas}
\begin{center}
  \begin{tabular}{cc|cc}
    \multicolumn{2}{c|}{Originales} & \multicolumn{2}{c}{Transformados} \\
      $x$ & $y$ & $x$ & $y$ \\ \hline
      0.0 & 0.0 & 0.5 & 0.5 \\ 
      4.0 & 7.0 & 2.0 & 3.5 \\ 
      5.0 & 3.0 & 2.5 & 1.5 \\ 
      3.0 & 5.0 & 1.5 & 2.5 \\ 
  \end{tabular}
\end{center}
\end{frame}

\begin{frame}[fragile]\frametitle{Multicolumnas}
\begin{block}{}
\begin{minted}{tex}
\begin{center}
  \begin{tabular}{cc|cc}
    \multicolumn{2}{c|}{Originales} & 
         \multicolumn{2}{c}{Transformados} \\
      $x$ & $y$ & $x$ & $y$ \\ \hline
      0.0 & 0.0 & 0.5 & 0.5 \\ 
      4.0 & 7.0 & 2.0 & 3.5 \\ 
      5.0 & 3.0 & 2.5 & 1.5 \\ 
      3.0 & 5.0 & 1.5 & 2.5 \\ 
  \end{tabular}
\end{center}
\end{minted}
\end{block}
\end{frame}

\begin{frame}[fragile]\frametitle{Elementos Flotantes}
En \LaTeX\ existen diversos tipos de \textbf{elementos flotantes}, cuya posición en el documento final es decidida al momento de compilar: \textit{tablas} y \textit{figuras}.
\begin{table}
	\begin{center}
    \begin{tabular}{c|cc}
      A\~no  & Ventas   & Inversión \\ \hline
      1999  & \$ 3.900 &    1.4\%   \\
      2000  & \$ 2.700 &    3.6\%   \\
      2001  & \$ 3.200 &    2.3\%   \\
      2002  & \$ 3.700 &    4.9\%   \\
      2003  & \$ 4.100 &    3.4\%   \\
    \end{tabular}
	\end{center}

	\caption{Ventas Empresa Pato Feliz}
	\label{tab:ventaspatofeliz}
\end{table}
\end{frame}

\begin{frame}[fragile]\frametitle{Elementos Flotantes}
\begin{block}{}
\begin{minted}{tex}
\begin{table}
  \begin{center}
    \begin{tabular}{c|cc}
      ...
    \end{tabular}
  \end{center}

  \caption{Ventas Empresa Pato Feliz}
  \label{tab:ventaspatofeliz}
\end{table}
\end{minted}
\end{block}
\end{frame}

\begin{frame}[fragile]\frametitle{Elementos Flotantes}
\begin{itemize}
  \item \LaTeX\ tratará de acomodar los elementos flotantes lo mejor que
    pueda en las páginas cercanas al código de la tabla.
  \item No tratar de forzar la posición de la tabla en el documento.
    \underline{\textit{Dejar que \LaTeX\ haga su trabajo}}.
  \item Utilizar \verb|\ref{..}| y \verb|\label{..}| para hacer referencia a la
    tabla. Evitar redacciones del tipo: ``\dots en el cuadro siguiente:''
\end{itemize}
\end{frame}

\begin{frame}[fragile]\frametitle{Insertar Figuras}
\begin{figure}
	\begin{center}
		\includegraphics[width=5cm]{figs/3cuerdas.pdf}
	\end{center}
	\caption{Un bloque sostenido por tres cuerdas.}
	\label{fig:3cuerdas}
\end{figure}

\end{frame}

\begin{frame}[fragile]\frametitle{Insertar Figuras}
\begin{block}{}
\begin{minted}{tex}
\usepackage{graphicx}

...

\begin{figure}
  \begin{center}
    \includegraphics[width=5cm]{3cuerdas.pdf}
  \end{center}
  \caption{Un bloque sostenido por tres cuerdas.}
  \label{fig:3cuerdas}
\end{figure}
\end{minted}
\end{block}
\end{frame}

\begin{frame}[fragile]\frametitle{Insertar Figuras}
\begin{itemize}
  \item (Cuando se generan archivos \texttt{.ps} (compilando con \texttt{latex}) se pueden insertar imágenes en formato \texttt{.eps}, \texttt{.ps}.)
  \item Cuando se generan archivos \texttt{.pdf} (compilando con \texttt{pdflatex}) se pueden insertar imágenes en formato \texttt{.jpg}, \texttt{.png}, \texttt{.pdf}.
  \item Recomiendo Inkscape, Python, LibreOffice  para crear \textit{gráficos vectoriales} (.svg, .ps, .eps, .pdf); Gimp para fotos (.png, .jpg).
\end{itemize}
\end{frame}

\begin{frame}[fragile]\frametitle{Insertar Figuras}
\begin{itemize}
\item La opción  \verb|[width=6cm]| se puede usar para modificar el ancho
  tama\~no de una imagen. Tambi'en existe la opción \texttt{height}, p.ej. \verb|[height=5cm]|.
\item Tambi'en puede usarse la opción \verb|[scale=0.6]| para re-escalar la figura.
\begin{verbatim}
	\includegraphics[scale=0.6]{transistor.pdf}
\end{verbatim}  
\end{itemize}
\end{frame}

\begin{frame}[fragile]\frametitle{'Indices}

\begin{itemize}
\item Los comandos \verb|\listoffigures| y \verb|\listoftables| generan
los 'indices de figuras y tablas respectivamente.
\item En los 'indices se agregan sólo las figuras y tablas que hayas agregado como
elementos flotantes.
\end{itemize}
\end{frame}


%
%\part{Definir nuevas macros}
%
%\begin{frame}[fragile]\frametitle{Nuevos Comandos}
%
%\begin{itemize}
%\item La instrucción 
%\begin{center}
%\verb|\newcommand{|\textit{comando}\verb|}{|\textit{definición}\verb|}|
%\end{center}
%se utiliza para definir nuevos comandos.
%\end{itemize}
%
%\begin{verbatim}
%\newcommand{\RR}{{\mathbb R}}    % Conjunto de los Reales
%\newcommand{\QQ}{{\mathbb Q}}    % Conjunto de los Racionales
%\newcommand{\tq}{\;|\;}
%\newcommand{\prove}{\vdash}
%\end{verbatim}
%\end{frame}
%
%\begin{frame}[fragile]\frametitle{Nuevos Comandos}
%\begin{itemize}
%\item Puedes tambi'en definir nuevos comandos con argumentos
%\begin{verbatim}[escapechar=\$]
%\newcommand{\set}[1]{\left\{#1\right\}}
%\newcommand{\iprod}[2]{\left\langle #1, #2 \right\rangle}
%\newcommand{\logic}[1]{\ensuremath{\mathrm{#1}}}
%\newcommand{\provein}[1]{\prove_\logic{#1}}
%\end{verbatim}
%\item El n'umero entre corchetes indica el n'umero de argumentos, y haces referencia
%a ellos con los comandos \verb|#1|, \verb|#2|, etc.
%\end{itemize}
%\end{frame}
%
%


\end{document}
